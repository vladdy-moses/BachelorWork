\section{Безопасность и экологичность проекта}

В эпоху активного развития информационных и компьютерных технологий возникает проблема сохранения благополучия и здоровья человека.
Из-за увеличения работы за персональной ЭВМ, а также последствий этого, будь то постоянный шум, сидячий режим работы или излишняя нагрузка на органы зрения, начали более активное развитие некоторые заболевания и отклонения в здоровье человека.

Охрана труда в современных реалиях ...

\subsection{Исходные данные}

\begin{footnotesize}
\begin{longtable}[h]{|p{0.05\textwidth}|p{0.35\textwidth}|p{0.5\textwidth}|}
	\caption{\label{tab:ecol-source}Исходные данные для проектирования} \\
	\hline
		\textbf{№№} &
		\textbf{Данные} &
		\textbf{Название} \\
	\hline \endfirsthead
	\caption*{Продолжение таблицы \ref{tab:ecol-source}}\\
	\hline
		\textbf{№№} &
		\textbf{Данные} &
		\textbf{Название} \\
	\hline \endhead
		1 & 
		Тема дипломного проекта &
		\WorkName \\
	\hline
		2 & 
		Технологический процесс &
		Единичный технологический процесс \\
	\hline
		3 & 
		Оборудование, в т. ч. паспортные данные &
		\begin{enumerate}
			\item Ноутбук Samsung NP350E5C-S06RU 
			\item Маршрутизатор NetGear WNR3500L v2
		\end{enumerate} \\
	\hline
		4 & 
		Персонал (состав, профессии) &
		1 программист, 1 сотрудник отдела контроля качества \\
	\hline
		5 & 
		Исходное состояние системы, ресурсы, материалы &
		Ресурсами является Интернет и доступ к внутренней сети ООО <<АИС Город>>.
		Материалами являются внутренние регламенты ООО <<АИС Город>>.\\
	\hline
		6 & 
		Энергоносители (электричество, вода, пар, газ, уголь) и их характеристики &
		Бытовая электросеть 220В. \\
	\hline
		7 & 
		Расположение рабочего места, функции персонала &
		Рабочее место программиста располагается в ФГБОУ ВПО УлГТУ.
		Программист разрабатывает ИС. \newline
		Рабочее место сотрудника отдела контроля качества располагается в офисе ООО АИС Город.
		Сотрудник отдела КК следит за соблюдением технического задания при разработке ИС. \\
	\hline
		8 & 
		Признаки отнесения объекта к опасным промышленным объектам &
		Отсутствуют. \\
	\hline
		9 & 
		Санитарная характеристика производства &
		Отсутствует. \\
	\hline
		10 & 
		Характеристика помещений по электроопасности &
		Помещения без повышенной опасности. \\
	\hline
		11 & 
		Характеристика среды помещений &
		Сухие помещения. \\
	\hline
		12 & 
		Категория производства по взрывопожарной опасности &
		Д --- пониженная пожароопасность. \\
	\hline
		13 & 
		Класс пожароопасной зоны &
		Отсутствует. \\
	\hline
		14 & 
		Класс взрывоопасной зоны &
		Отсутствует. \\
	\hline
		15 & 
		Рассматриваемые стадии <<жизненного цикла>> продукции &
		\begin{enumerate}
			\item Формирование требований
			\item Разработка концепции АС 
			\item Техническое задание 
			\item Эскизный проект
			\item Технический проект
			\item Рабочая документация
			\item Тестирование
			\item Ввод в действие
			\item Сопровождение
		\end{enumerate} \\
	\hline
		16 & 
		Классы условий труда в соответствии с Картой аттестации рабочего места по условиям труда: \newline
		по вредности, \newline
		по травмоопасности &
		По вредности --- вредный (III) класс. \newline
		По травмоопасности --- допустимый (II) класс. \\
	\hline
		17 & 
		Вредные и опасные производственные факторы &
		ы \\
	\hline
		18 & 
		Виды загрязнений окружающей среды &
		Отсутствуют. \\
	\hline
		19 & 
		Возможные чрезвычайные ситуации &
		ы \\
	\hline
\end{longtable}
\end{footnotesize}

\subsection{Перечень нормативных документов}

\begin{enumerate}
	\item Санитарные правила и нормы. СанПиН 2.2.2./2.4.1340-03 Гигиенические требования к персональным электронно-вычислительным машинам и организации работы. 
	\item <<Руководство по гигиенической оценке факторов рабочей среды  и трудовых процессов. Критерии и классификация условий труда>>. Р 2.2.2006-05.
	\item ГОСТ 12.0.003-74.ССБТ. (СТ СЭВ 790-77) Опасные и вредные производственные факторы. Классификация. М.: Изд-во стандартов, 1996.
	\item ГОСТ 12.1.003-83.ССБТ. Шум. Общие требования безопасности. М.: Изд-во стандартов.1996.
	\item ГОСТ 12.1.004-91.ССБТ. Пожарная безопасность. Общие требования. М.: Изд-во стандартов, 1996.
	\item ГОСТ 12.1.005-88.ССБТ. Общие санитарно-гигиенические требования к воздуху рабочей зоны. М.: Изд-во стандартов, 1996.
	\item ГОСТ 12.1.006-88.ССБТ. Электромагнитные поля  радиочастот. Допустимые уровни на рабочих местах и требования к проведения контроля. М.: Изд-во стандартов, 1998.
	\item ГОСТ 12.1.019-79.ССБТ (СТ СЭВ 4880-84). Электробезопасность. Общие требования. М.: Изд-во стандартов, 1996.
	\item ГОСТ 12.1.030-81.ССБТ. Электробезопасность. Защитное заземление зануление. М.: Изд-во стандартов, 1996.
	\item ГОСТ 12.1.038-82.ССБТ. Электробезопасность. Предельно-допустимые значения напряжений прикосновения и токов. М.: Изд-во стандартов, 1996.
	\item Правила устройства электроустановок. М.: Энергия, 1987.
	\item Общесоюзные нормы технологического проектирования ОНТП 24-86., М.: МВД СССР, 1986.
	\item СНиП 2.01.02-85. Противопожарные нормы. М.: Стройиздат,1986.
	\item СНиП 2.04.05-86. Отопление, вентиляция, кондиционирование возду-ха. М.: Стройиздат, 1988.
	\item СНиП 23-05-95. Естественное и искусственное освещение. Анализ проектирования. М.: Энерго, 1996.
	\item Р 2.2.013-94. Гигиена труда. М.: Госкомсанэпиднадзор России, 1994.
	\item Правила пожарной безопасности в Российской Федерации – ППБ 01 03. 
	\item Нормы пожарной безопасности – НПБ 88-2001. Установки пожаротушения и сигнализации. Нормы и правила проектирования.
\end{enumerate}

\subsection{Анализ потенциально опасных факторов}

Work in process.

\subsubsection{Анализ вредных и опасных производственных факторов}

Work in process.

\subsubsection{Анализ воздействия на окружающую среду}

Work in process.

\subsubsection{Анализ возможных чрезвычайных ситуаций}

Work in process.

\subsubsection{Обоснование расчетной части}

Work in process.

\subsection{Мероприятия по охране труда}

Work in process.

\subsubsection{Мероприятия по обеспечению комфортных условий труда}

Work in process.

\subsubsection{Мероприятия по защите от опасных производственных факторов}

Work in process.

\subsubsection{Мероприятия по защите от вредных производственных факторов}

Work in process.

\subsubsection{Квалификационные требования к персоналу}

Work in process.

\subsection{Мероприятия по охране окружающей среды}

Work in process.

\subsection{Мероприятия по защите от чрезвычайных ситуаций}

Work in process.

\subsection{Расчетная часть}

Work in process.

\subsection{Оценка эффективности принятых решений}

Work in process.

\clearpage
\newpage