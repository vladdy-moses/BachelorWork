\section{Безопасность и экологичность проекта}

В эпоху активного развития информационных и компьютерных технологий возникает проблема сохранения благополучия и здоровья человека.
Из-за увеличения работы за персональной ЭВМ, а также последствий этого, будь то постоянный шум, сидячий режим работы или излишняя нагрузка на органы зрения, начали более активное развитие некоторые заболевания и отклонения в здоровье человека.

Охрана труда в современных реалиях ...

\subsection{Исходные данные}

\begin{footnotesize}
\begin{longtable}[h]{|p{0.05\textwidth}|p{0.35\textwidth}|p{0.5\textwidth}|}
	\caption{\label{tab:ecol-source}Исходные данные для проектирования} \\
	\hline
		\textbf{№№} &
		\textbf{Данные} &
		\textbf{Название} \\
	\hline \endfirsthead
	\caption*{Продолжение таблицы \ref{tab:ecol-source}}\\
	\hline
		\textbf{№№} &
		\textbf{Данные} &
		\textbf{Название} \\
	\hline \endhead
		1 & 
		Тема дипломного проекта &
		\WorkName \\
	\hline
		2 & 
		Технологический процесс &
		Единичный технологический процесс \\
	\hline
		3 & 
		Оборудование, в т. ч. паспортные данные &
		\begin{enumerate}
			\item Ноутбук Samsung NP350E5C-S06RU 
			\item Маршрутизатор NetGear WNR3500L v2
		\end{enumerate} \\
	\hline
		4 & 
		Персонал (состав, профессии) &
		1 программист, 1 сотрудник отдела контроля качества \\
	\hline
		5 & 
		Исходное состояние системы, ресурсы, материалы &
		Ресурсами является Интернет и доступ к внутренней сети ООО <<АИС Город>>.
		Материалами являются внутренние регламенты ООО <<АИС Город>>.\\
	\hline
		6 & 
		Энергоносители (электричество, вода, пар, газ, уголь) и их характеристики &
		Бытовая электросеть 220В. \\
	\hline
		7 & 
		Расположение рабочего места, функции персонала &
		Рабочее место программиста располагается в ФГБОУ ВПО УлГТУ.
		Программист разрабатывает ИС. \newline
		Рабочее место сотрудника отдела контроля качества располагается в офисе ООО АИС Город.
		Сотрудник отдела КК следит за соблюдением технического задания при разработке ИС. \\
	\hline
		8 & 
		Признаки отнесения объекта к опасным промышленным объектам &
		Отсутствуют. \\
	\hline
		9 & 
		Санитарная характеристика производства &
		Отсутствует. \\
	\hline
		10 & 
		Характеристика помещений по электроопасности &
		Помещения без повышенной опасности. \\
	\hline
		11 & 
		Характеристика среды помещений &
		Сухие помещения. \\
	\hline
		12 & 
		Категория производства по взрывопожарной опасности &
		Д --- пониженная пожароопасность. \\
	\hline
		13 & 
		Класс пожароопасной зоны &
		Отсутствует. \\
	\hline
		14 & 
		Класс взрывоопасной зоны &
		Отсутствует. \\
	\hline
		15 & 
		Рассматриваемые стадии <<жизненного цикла>> продукции &
		\begin{enumerate}
			\item Формирование требований
			\item Разработка концепции АС 
			\item Техническое задание 
			\item Эскизный проект
			\item Технический проект
			\item Рабочая документация
			\item Тестирование
			\item Ввод в действие
			\item Сопровождение
		\end{enumerate} \\
	\hline
		16 & 
		Классы условий труда в соответствии с Картой аттестации рабочего места по условиям труда: \newline
		по вредности, \newline
		по травмоопасности &
		По вредности --- вредный (III) класс. \newline
		По травмоопасности --- допустимый (II) класс. \\
	\hline
		17 & 
		Вредные и опасные производственные факторы &
		ы \\
	\hline
		18 & 
		Виды загрязнений окружающей среды &
		Отсутствуют. \\
	\hline
		19 & 
		Возможные чрезвычайные ситуации &
		ы \\
	\hline
\end{longtable}
\end{footnotesize}

\subsection{Перечень нормативных документов}

\begin{enumerate}[1.]
	\item Санитарные правила и нормы. СанПиН 2.2.2./2.4.1340-03 Гигиенические требования к персональным электронно-вычислительным машинам и организации работы. 
	\item <<Руководство по гигиенической оценке факторов рабочей среды  и трудовых процессов. Критерии и классификация условий труда>>. Р 2.2.2006-05.
	\item ГОСТ 12.0.003-74.ССБТ. (СТ СЭВ 790-77) Опасные и вредные производственные факторы. Классификация. М.: Изд-во стандартов, 1996.
	\item ГОСТ 12.1.003-83.ССБТ. Шум. Общие требования безопасности. М.: Изд-во стандартов.1996.
	\item ГОСТ 12.1.004-91.ССБТ. Пожарная безопасность. Общие требования. М.: Изд-во стандартов, 1996.
	\item ГОСТ 12.1.005-88.ССБТ. Общие санитарно-гигиенические требования к воздуху рабочей зоны. М.: Изд-во стандартов, 1996.
	\item ГОСТ 12.1.006-88.ССБТ. Электромагнитные поля  радиочастот. Допустимые уровни на рабочих местах и требования к проведения контроля. М.: Изд-во стандартов, 1998.
	\item ГОСТ 12.1.019-79.ССБТ (СТ СЭВ 4880-84). Электробезопасность. Общие требования. М.: Изд-во стандартов, 1996.
	\item ГОСТ 12.1.030-81.ССБТ. Электробезопасность. Защитное заземление зануление. М.: Изд-во стандартов, 1996.
	\item ГОСТ 12.1.038-82.ССБТ. Электробезопасность. Предельно-допустимые значения напряжений прикосновения и токов. М.: Изд-во стандартов, 1996.
	\item Правила устройства электроустановок. М.: Энергия, 1987.
	\item Общесоюзные нормы технологического проектирования ОНТП 24-86., М.: МВД СССР, 1986.
	\item СНиП 2.01.02-85. Противопожарные нормы. М.: Стройиздат,1986.
	\item СНиП 2.04.05-86. Отопление, вентиляция, кондиционирование возду-ха. М.: Стройиздат, 1988.
	\item СНиП 23-05-95. Естественное и искусственное освещение. Анализ проектирования. М.: Энерго, 1996.
	\item Р 2.2.013-94. Гигиена труда. М.: Госкомсанэпиднадзор России, 1994.
	\item Правила пожарной безопасности в Российской Федерации – ППБ 01 03. 
	\item Нормы пожарной безопасности – НПБ 88-2001. Установки пожаротушения и сигнализации. Нормы и правила проектирования.
\end{enumerate}

\subsection{Анализ потенциально опасных факторов}

На рис.~\ref{img:ecology-schema}, приведена принципиальная блок-схема обеспечения безопасности объекта проектирования.

\begin{figure}[h!]
	\begin{center}
		\begin{minipage}[h]{\linewidth}
			\includegraphics[width=\linewidth]{images/ecology-schema.png}
			\caption{Принципиальная блок-схема обеспечения безопасности объекта проектирования}
			\label{img:ecology-schema}
		\end{minipage}
		\hfill
	\end{center}
\end{figure}  

\subsubsection{Анализ вредных и опасных производственных факторов}

Опасный производственный фактор --- это производственный фактор, воздействие которого в определенных условиях приводит к травме или к другому внезапному ухудшению здоровья. 

Воздействие вредного производственного фактора в определенных условиях приводит к заболеванию или снижению работоспособности.

Классификация опасных и вредных производственных факторов (ГОСТ 12.0.003-74).
Опасные и вредные производственные факторы подразделяются по природе действия на следующие группы:

\begin{enumerate}
	\item физические;
	\item химические;
	\item биологические;
	\item психофизические.
\end{enumerate}

Все факторы, за исключением психофизических обусловлены воздействием техники и рабочей среды.
Психофизиологические факторы связаны с влиянием тяжести и напряженности труда, что в конечном итоге тоже может привести к заболеваниям.

Так как на рабочем месте, рассматриваемом в рамках данного дипломного проекта, химические и биологические опасные и вредные производственные факторы оказывают незначительное, по сравнению с  физическими факторами, влияние, в рассмотрение они браться не будут.

При работе с ПЭВМ на пользователя в той или иной степени могут воздействовать следующие физические факторы: повышенные уровни переменного электромагнитного и электростатического полей; повышенный уровень статического электричества; повышенный уровень низкоэнергетического (мягкого) рентгеновского ионизирующего излучения; повышенные уровни ультрафиолетового и инфракрасного излучения; повышенное содержание положительных аэроионов в воздухе рабочей зоны; пониженное содержание отрицательных аэроионов; аномальный уровень освещённости рабочей зоны; повышенная яркость фрагментов светового изображения или света, попадающего в поле зрения пользователя; по-вышенная неравномерность распределения яркости в поле зрения пользователя; повышенная внешняя освещённость экрана; повышенные пульсации светового потока источников света или светового потока, излучаемого экраном; неблагоприятный для работы спектр излучения источников света; повышенная временная нестабильность изображения; мерцание экрана; изменение яркости свечения экрана; повышенная прямая блескость, вызванная попаданием в поле зрения работающего чрезмерно яркого света различных излучающих объектов; повышенная отражённая блескость, обусловленная наличием зеркальных отражений (бликов), в том числе от экрана; повышенный уровень шума; аномальные температура, влажность и подвижность воздуха рабочей зоны; повышенное значение напряжения в электрической цепи, замыкание которой может произойти через тело человека; пожар.

\point Шум

Шум является общебиологическим раздражителем и в определенных условиях может влиять на все органы и системы организма человека.
Кроме непосредственного воздействия на орган слуха шум влияет на различные отделы головного мозга, изменяя нормальные процессы высшей нервной деятельности.
Это так называемое неспецифическое воздействие шума может возникнуть даже раньше, чем изменения в органе слуха.
Шумовые явления обладают свойством аккумуляции: накапливаясь в организме, он все больше и больше угнетает нервную систему.
Шум -- причина преждевременного утомления, ослабления внимания, памяти.

Согласно СанПиН 2.2.2./2.4.1340-03, допустимым уровнем звукового давления при работе на ВДТ и ПЭВМ не должно превышать 60 дБ.
Экспериментальные данные показывают, что уровень звукового давления (33 дБ) меньше предельно допустимого уровня.

Мероприятия по защите от шума, проводимые в производственном помещении соответствуют ГОСТ 12.1.003-83 и других мероприятий по улучшению шумовой обстановки не требуется.

\point Микроклимат

Микроклимат помещений - это климат внутренней среды помещений, который определяется действующими на организм человека сочетаниями температуры, влажности и скорости движения воздуха, а также температуры окружающих поверхностей.

Показателями, характеризующими микроклимат в производственном помещении, являются:

\begin{enumerate}
	\item температура воздуха;
	\item относительная влажность воздуха;
	\item скорость движения воздуха;
	\item интенсивность теплового излучения.
\end{enumerate}

Значительное отклонение микроклимата рабочей зоны от оптимального может быть причиной ряда физиологических нарушений в организме работающих, привести к резкому снижению работоспособности и даже к профессиональным заболеваниям.

В помещениях с вычислительной техникой при выполнении работ операторского типа, связанных с нервно-эмоциональным напряжением, по ГОСТ 12.1.005-88 необходимо соблюдать оптимальные величины показателей:

\begin{enumerate}
	\item температура помещения в переходный период $18-22^{\circ}C$, в холодный период $22-24^{\circ}C$, в теплый период $20-24^{\circ}C$;
	\item подвижность воздуха – от 0,1 до 0,2 м/с;
	\item влажность воздуха составляет 60--70\%;
	\item воздействие химических веществ отсутствует;
	\item запыленности и загазованности воздуха нет;
	\item выполняются легкие физические работы (1 категория).
\end{enumerate}

Колебания температуры воздуха допускаются до 4\%.

Для создания нормальных условий труда в производственных помещениях обеспечивают нормативные значения параметров микроклимата --- температуры воздуха, относительную влажность и скорость движения, а также интенсивности теплового излучения.

В ГОСТ 12.1.005-88 указаны оптимальные и допустимые показатели микроклимата в производственных помещениях.
Оптимальные показатели распространяются на всю рабочую зону, а допустимые устанавливают раздельно для постоянных и непостоянных рабочих мест в тех случаях, когда по технологическим техническим или экономическим причинам невозможно обеспечить оптимальные нормы.

Мероприятия по обеспечению оптимальных метеоусловий соответствуют ГОСТ 12.1.005-88 и СНиП 2.04.05-86 и других мероприятий по обеспечению микроклимата не требуется.

\point Электрический ток

Опасное и вредное воздействие на людей электрического тока проявляется в виде электротравм и профессиональных заболеваний.
Степень опасного и вредного воздействий на человека электрического тока зависит от:

\begin{enumerate}
	\item рода и величины напряжения и тока; 
	\item частоты электрического тока;
	\item пути прохождения тока через тело человека (наибольшая опасность воз-никает при непосредственном прохождении тока через жизненно важные органы);
	\item продолжительности воздействия на организм человека (с течением времени резко падает сопротивление кожи человека, более вероятным становится поражение сердца, и накапливаются другие отрицательные последствия);
	\item условий внешней среды.
\end{enumerate}

Согласно ГОСТ 12.1.038-82, человек начинает ощущать протекающий через него ток в 0,3 мА (50 Гц), 0,4 мА (400 Гц) и 1 мА (постоянный).
Это пороговый ощутимый ток.
Ток 10 – 15 мА (50 Гц) называется пороговым не отпускающим.
Он вызывает судорожные сокращения мышц руки, в которой зажат проводник.
Ток 25 – 50 мА (50 Гц) приводит к затруднению и даже прекращению дыхания, а при 100 мА ток вызывает остановку или фибрилляцию сердца (хаотические и разновременные сокращения волокон сердечной мышцы, полностью нарушающие ее работу как насоса), прекращению кровообращения и смерть.
При постоянном токе пороговый не отпускающий ток 50 – 70 мА, а фибрилляционный – до 0,3 А.

Существует два вида электротравм:

Электрические удары --- это возбуждение живых тканей организма протекающим через него электрическим током, проявляющееся в непроизвольных судорожных сокращениях различных мышц тела.
В результате электрического удара могут возникнуть или обостриться сердечно-сосудистые заболевания, а также нервные болезни.
Нередко появляется рассеянность, ослабевают память и внимание.

Электрические травмы --- это поражение внешних частей тела человека, к ним относятся: электрический ожог, электрометаллизация кожи и электрические знаки тока.

Причинами смерти от электрического тока могут быть прекращение работы сердца, остановка дыхания и электрический шок.

\point Электромагнитное и ионизирующее излучение

Электромагнитным излучением называется излучение, прямо или косвенно вызывающее ионизацию среды.
Контакт с электромагнитными излучениями представляет серьезную опасность для человека.

Основным источником электромагнитного излучения при работе с ПЭВМ является монитор.
Дисплей излучает электромагнитные поля (ЭМП) в очень широком диапазоне частот (от 3 Гц до 300 мГц), но преобладают следующие два диапазона:

\begin{enumerate}
	\item поля, создаваемые блоком сетевого питания и блоком кадровой развертки дисплея (например, с частотой 50–150 Гц – электромагнитные поля от блока питания, проводов и системы вертикального отклонения и модуляции луча ЭЛТ); основной энергетический спектр этих полей сосредоточен в диапазоне частот до 1 кГц;
	\item поля, создаваемые блоком строчной развертки и блоком сетевого питания ПЭВМ (если он импульсный); основной энергетический спектр этих полей сосредоточен в диапазоне частот от 15 до 100 кГц.
\end{enumerate}

Защита от электромагнитного излучения компьютера: 

\begin{enumerate}[1.]
	\item По возможности, стоит приобрести жидкокристаллический монитор, поскольку его излучение значительно меньше, чем у распространённых ЭЛТ мониторов (монитор с электроннолучевой трубкой). 
	\item Системный блок и монитор должен находиться как можно дальше от человека.
	\item Не оставлять компьютер включённым на длительное время, если он не используется, например, использовать "спящий режим" для монитора.
	\item В связи с тем, что электромагнитное излучение от стенок монитора намного больше, лучше постараться поставить монитор в угол, так что бы излучение поглощалось стенами. Особое внимание стоит обратить на расстановку мониторов в офисах.
	\item По возможности сократить время работы за компьютером и чаще прерывать работу.
	\item Компьютер должен быть заземлён. Если есть защитный экран, то его тоже следует заземлить, для этого специально предусмотрен провод на конце которого находиться металлическая прищепка (не цепляйте её к системному блоку)
\end{enumerate}

Ионизирующее излучение --- это любое излучение, вызывающее ионизацию среды, т.е. протекание электрических токов в этой среде, в том числе и в организме человека, что часто приводит к разрушению клеток, изменению состава крови, ожогам и другим тяжелым последствиям.

Излучения на расстоянии 40 см от экрана составляют около 0.08 мкР/ч, что не превышает нормы.
И по данному фактору можно отнести работы с персональным компьютером к допустимым по степени вредности. 

Исходя из вышесказанного, условия работы с персональным компьютером удовлетворяют требованиям Р 2.2.013-94 и СанПиН 2.2.2./2.4.1340 03, но необходимы дополнительные меры защиты в виде регламентирования рабочего времени.

\point Освещённость

Правильно спроектированное и выполненное производственное освещение улучшает условия зрительной работы, снижает утомляемость, способствует по-вышению производительности труда, благотворно влияет на производственную среду, оказывая положительное психологическое воздействие на работающего, повышает безопасность труда и  снижает травматизм.

Недостаточность освещения приводит к напряжению зрения, ослабляет внимание, приводит к наступлению преждевременной утомленности. Чрезмерно яркое освещение вызывает ослепление, раздражение и резь в глазах. Неправильное направление света на рабочем месте может создавать резкие тени, блики, дезориентировать работающего. Все эти причины могут привести к несчастному случаю или профзаболеваниям, поэтому столь важен правильный расчет освещенности.

Существует три вида освещения - естественное, искусственное и совмещен-ное (естественное и искусственное вместе).

Естественное освещение --- освещение помещений дневным светом, проника-ющим через световые проемы в наружных ограждающих конструкциях помещений.

Естественное освещение характеризуется тем, что меняется в широких пре-делах в зависимости от времени  дня, времени года, характера области и ряда других факторов.

Искусственное освещение применяется при работе в темное время суток и днем, когда не удается обеспечить нормированные значения коэффициента естественного освещения (пасмурная погода, короткий световой день).

Освещение, при котором недостаточное по нормам естественное освещение дополняется искусственным, называется совмещенным освещением.

Искусственное освещение подразделяется на рабочее, аварийное, эвакуационное, охранное. Рабочее освещение, в свою очередь, может быть общим или комбинированным. Общее - освещение, при котором светильники размещаются в верхней зоне помещения равномерно или применительно к расположению оборудования. Комбинированное - освещение, при котором к общему добавляется местное освещение.

Согласно СНиП II-4-79 в помещений вычислительных центров необходимо применить систему комбинированного освещения.

При выполнении работ категории высокой зрительной точности (наимень-ший размер объекта различения 0,3.0,5мм) величина коэффициента естественного освещения (КЕО) должна быть не ниже 1,5%, а при зрительной работе средней точности (наименьший размер объекта различения 0,5.1,0 мм) КЕО должен быть не ниже 1,0%. В качестве источников искусственного освещения обычно используются люминесцентные лампы типа ЛБ или ДРЛ, которые попарно объединяются в светильники, которые должны располагаться над рабочими поверхностями равномерно.

Требования к освещенности в помещениях, где установлены компьютеры, следующие: при выполнении зрительных работ высокой точности общая осве-щенность должна составлять 300лк, а комбинированная - 750лк; аналогичные требования при выполнении работ средней точности - 200 и 300лк соответственно.

Кроме того все поле зрения должно быть освещено достаточно равномерно – это основное гигиеническое требование. Иными словами, степень освещения помещения и яркость экрана компьютера должны быть примерно одинаковыми, т.к. яркий свет в районе периферийного зрения значительно увеличивает напряженность глаз и, как следствие, приводит к их быстрой утомляемости.

\subsubsection{Анализ воздействия на окружающую среду}

В жизненном цикле компьютерной техники можно выделить три этапа: производство, эксплуатация, утилизация.

Вопросы защиты окружающей среды в процессе производства компьютеров возникли давно и регламентируются сейчас, в частности стандартом ТСО-03 NUTEC, по которому контролируются выбросы токсичных веществ, условия работы и др.
Согласно ТСО-03 произведенное оборудование может быть сертифицировано лишь в том случае, если не только контролируемые параметры самого оборудования соответствуют требованиям этого стандарта, но и технология производства этого оборудования отвечает требованиям стандарта.

Воздействие компьютеров на окружающую среду при эксплуатации регламентировано рядом стандартов.
Выделяют две группы стандартов и рекомендаций: по безопасности и эргономике.

При утилизации старых компьютеров происходит их разработка на фракции: металлы, пластмассы, стекло, провода, штекеры.
Из одной тонны компьютерного лома получают до 200 кг меди, 480 кг железа и нержавеющей стали, 32 кг алюминия, 3 кг серебра, 1 кг золота и 300 г палладия.

Переработку промышленных отходов производят на специальных полигонах, создаваемых в соответствии с требованиями СНиП 2.01.28-85 и предназначенных для централизованного сбора обезвреживания и захоронения токсичных отходов промышленных предприятий, НИИ и учреждений.

\subsubsection{Анализ возможных чрезвычайных ситуаций}

Чрезвычайная ситуация (ЧС) – это обстановка на определенной территории, сложившаяся в результате аварии, опасного природного явления, катастрофы, стихийного или иного бедствия, которые могут повлечь или повлекли за собой человеческие жертвы, ущерб здоровью людей или окружающей природной среде, значительные материальные потери и нарушение условий жизнедеятельности людей.

ЧС являются многофакторными событиями, которые могут возникать в результате многочисленных причин, в различных условиях и приводить к разнообразным последствиям.

По происхождению ЧС подразделяются на природные, техногенные, антропогенные, военные.

Под техногенной ЧС понимается состояние, при котором в результате возникновения источника техногенной ЧС на объекте, определенной территории или акватории нарушаются нормальные условия жизни и деятельности людей, возникает угроза их жизни и здоровью, наносится ущерб имуществу населения, народному хозяйству и окружающей среде (ГОСТ 22.0.05-94).

Авария --- опасное техногенное происшествие, создающее на объекте, определенной территории или акватории угрозу жизни и здоровью людей и приводящее к разрушению зданий, сооружений, оборудования и транспортных средств, нарушению транспортного или производственного процесса, а также нанесению ущерба окружающей природной среде (ГОСТ 22.0.05-94). Крупная авария, как правило с человеческими жертвами, является катастрофой.

В соответствии с Постановлением Правительства РФ <<О классификации чрезвычайных ситуаций природного и техногенного характера>> (1996) ЧС подразделяются в зависимости от показателей:

• количество людей, пострадавших в ЧС;
• количество людей, у которых оказались нарушены условия жизнедеятельности;
• размер материального ущерба;
• размер зоны распространения поражающих факторов (рис. 13).

При идентификации возможных техногенных ЧС, связанных с объектом проектирования, необходимо провести их анализ в зависимости от происхождения, масштаба распространения, вида поражающих факторов Так, например, для котельной возможными ЧС являются пожар, взрыв, вызванные воспламенением газа, мазута; разгерметизация систем, работающих под давлением, и воздействие рабочих сред на человека, а аварии в системах электроснабжения приведут к потере их устойчивости.

Существует ряд отраслей производства, которые, в случае возникновения на них аварий, могут создавать наиболее опасные ситуации. Они относятся к опасным производственным объектам. 

Из анализа промышленных аварий и катастроф следует, что причинами ЧС зачастую являются ошибки при проектировании и недостаточный уровень современных знаний.

Анализ потенциально опасных факторов, связанных с проектируемым объектом, должен явиться основой для обоснования необходимости расчета защиты от наиболее опасного фактора.

К техногенным относят ЧС, происхождение которых связано с техническими объектами, — пожары, взрывы, аварии на химически опасных объектах, выбросы радиоактивных веществ, обрушение зданий, аварии на системах жизнеобеспечения.

К природным относятся ЧС, связанные с проявлением стихийных сил природы, — землетрясения, наводнения, извержения вулканов, оползни, сели, ураганы, смерчи, бури, природные пожары и др.

К экологическим ЧС относятся аномальное природное загрязнение атмо-сферы, разрушение озонового слоя земли, опустынивание земель, засоление почв, кислотные дожди и др.

К биологическим ЧС относятся эпидемии, эпизоотии, эпифитотии.

К социальным ЧС относятся события, происходящие в обществе, — межнациональные конфликты, терроризм, грабежи, геноцид, войны и др.

Антропогенные ЧС являются следствием ошибочных действий людей.

Чрезвычайные ситуации классифицируются в зависимости от количества людей, пострадавших в этих ситуациях, людей, у которых оказались нарушены условия жизнедеятельности, от размера материального ущерба, а также границы зон распространения поражающих факторов чрезвычайной ситуации.

Анализ чрезвычайных ситуаций, имевших место в России за последние годы, позволил выделить причины аварийности и травматизма:

•	человеческий фактор — 50,1%;
•	оборудование, техника — 18,1%;
•	технология выполнения работ — 7,8%;
•	условия внешней среды — 16,6%;
•	прочие факторы — 7,4%.

В настоящее время заметно возрос удельный вес аварий, происходящих из-за неправильных действий обслуживающего технического персонала (более 50%). Часто это связано с недостаточностью профессионализма, а также неумением принимать оптимальные решения в сложной критической обстановке в условиях дефицита времени.

\subsubsection{Обоснование расчетной части}

Work in process.

\subsection{Мероприятия по охране труда}

Охрана труда– это система сохранения жизни и здоровья работников в процессе трудовой деятельности, включающая в себя правовые, социально-экономические, организационно-технические, санитарно-гигиенические, лечебно-профилактические, реабилитационные и иные мероприятия.

Условно охрану труда (ОТ) можно представить совокупностью четырех составляющих:

• правовая охрана труда (ПОТ);
• техника безопасности (ТБ);
• производственная санитария (ПС);
• пожарная безопасность (ПБ).
В соответствии со ст. 210 ТК РФ основными направлениями государственной политики в области охраны труда являются:
• обеспечение приоритета сохранения жизни и здоровья работников;
• принятие и реализация федеральных законов и иных нормативных правовых актов Российской Федерации, законов и иных нормативных правовых актов субъектов Российской Федерации в области охраны труда, а также федеральных целевых, ведомственных целевых и территориальных целевых программ улучшения условий и охраны труда;
• государственное управление охраной труда;
• государственный надзор и контроль за соблюдением государственных нормативных требований охраны труда;
• государственная экспертиза условий труда;
• установление порядка проведения аттестации рабочих мест по условиям труда и порядка подтверждения соответствия организации работ по охране труда государственным нормативным требованиям охраны труда;
• содействие общественному контролю за соблюдением прав и законных интересов работников в области охраны труда;
• профилактика несчастных случаев и повреждения здоровья работников;
• расследование и учет несчастных случаев на производстве и профессио-нальных заболеваний;
• защита законных интересов работников, пострадавших от несчастных случаев на производстве и профессиональных заболеваний, а также членов их семей, на основе обязательного социального страхования работников от несчастных случаев на производстве и профессиональных заболеваний;
• установление компенсаций за тяжелую работу и работу с вредными и (или) опасными условиями труда;
• координация деятельности в области охраны труда, охраны окружающей природной среды и других видов экономической и социальной деятельности;
• распространение передового отечественного и зарубежного опыта работы по улучшению условий и охраны труда;
• участие государства в финансировании мероприятий по охране труда;
• подготовка специалистов по охране труда и повышение их квалификации;
• организация государственной статистической отчетности об условиях труда, а также о производственном травматизме, профессиональной заболеваемости и об их материальных последствиях;
• обеспечение функционирования единой информационной системы охраны труда;
• международное сотрудничество в области охраны труда;
• проведение эффективной налоговой политики, стимулирующей создание безопасных условий труда, разработку и внедрение безопасных техники и технологий, производство средств индивидуальной и коллективной защиты работников;
• установление порядка обеспечения работников средствами индивидуальной и коллективной защиты, а также санитарно-бытовыми помещениями и устройствами, лечебно-профилактическими средствами.

Производственные процессы должны быть пожаро- и взрывобезопасными.

Производственные процессы не должны загрязнять окружающую среду (воздух, почву, водоемы) выбросами вредных веществ.

\subsubsection{Мероприятия по обеспечению комфортных условий труда}

В целях предотвращения неблагоприятного влияния на здоровье работников вредных факторов производственной среды и трудового процесса при исполь-зовании ими персональных электронно-вычислительных машин (ПЭВМ) режим их работы рекомендовано устанавливать в зависимости от вида и категории трудовой деятельности.

Виды трудовой деятельности разделяются на три группы: группа А - работа по считыванию информации с монитора компьютера с предварительным запросом; группа Б - работа по вводу информации; группа В - творческая работа в режиме диалога с ПЭВМ. При выполнении в течение рабочей смены работ, относящихся к разным видам трудовой деятельности, за основную работу с ПЭВМ следует принимать такую, которая занимает не менее 50% времени в течение рабочей смены или рабочего дня.

Для видов трудовой деятельности устанавливаются три категории тяжести и напряженности работы с ПЭВМ, которые определяются: для группы А - по суммарному числу считываемых знаков за рабочую смену, но не более 60000 знаков за смену; для группы Б - по суммарному числу считываемых или вводимых знаков за рабочую смену, но не более 40000 знаков за смену; для группы В - по суммарному времени непосредственной работы с ПЭВМ за рабочую смену, но не более 6 часов за смену.

В зависимости от категории трудовой деятельности и уровня нагрузки за рабочую смену при работе с ПЭВМ устанавливается суммарное время регламентированных перерывов.

Для предупреждения преждевременной утомляемости пользователей ПЭВМ рекомендуется организовывать рабочую смену путем чередования работ с использованием ПЭВМ и без него.

При возникновении у работающих с ПЭВМ зрительного дискомфорта и других неблагоприятных субъективных ощущений, несмотря на соблюдение санитарно-гигиенических и эргономических требований, рекомендуется применять индивидуальный подход с ограничением времени работы с ПЭВМ.

В случаях, когда характер работы требует постоянного взаимодействия с монитором компьютера (набор текстов или ввод данных и т.п.) с напряжением внимания и сосредоточенности, при исключении возможности периодического переключения на другие виды трудовой деятельности, не связанные с ПЭВМ, рекомендуется организация перерывов на 10 - 15 мин. через каждые 45 - 60 мин. работы.

Продолжительность непрерывной работы с ПК без регламентированного перерыва не должна превышать одного часа.

При работе с ПЭВМ в ночную смену (с 22 до 6 часов) независимо от категории и вида трудовой деятельности продолжительность регламентированных перерывов следует увеличивать на 30%.

Существует множество превентивных (предупредительных) мероприятий, позволяющих повысить безопасность работы. Одна из них заключается в создании на рабочем месте соответствующего инженерного обеспечения. Задача – сделать работу более комфортабельной, менее утомительной, помочь работнику стать более бдительным, менее открытым для несчастных случаев.

Работающим на ПЭВМ с высоким уровнем напряженности во время регламентированных перерывов и в конце рабочего дня рекомендуется посещать специально оборудованные комнаты для снятия напряжения.

\subsubsection{Мероприятия по защите от опасных и вредных производственных факторов}

Задачей защиты человека от опасных вредных производственных факторов (ОВПФ) является снижение уровня вредных факторов, не превышающих ПДУ и ПДК и риска появления опасных факторов до величин приемлемого риска.

Основные мероприятия по защите человека от опасных и вредных произ-водственных факторов приведены ниже:

1. Совершенствование технологии производств и технических средств с целью снижения уровня ОВПФ.
2. Защита расстоянием (удаление от источника ОВПФ).
3. Защита временем (уменьшение времени пребывания в зоне действия ОВПФ).
4. Применение средств защиты:
а) применение средств коллективной защиты;
б) применение средств индивидуальной защиты.

Защита человека от физических негативных факторов осуществляется тремя основными методами:

1) ограничение времени пребывания в зоне действия физического поля;
2) удаление от источника поля;
3) применение средств защиты.

Для защиты от акустических колебаний (шума, ультра и инфразвука) проводят следующие мероприятия:

1) снижение звуковой мощности источника звука;
2) размещение рабочих мест с учетом направленности излучения от источника звука;
3) акустическая обработка помещений (применение звукопоглощения облицовки, штучные, объемные поглотители различных конструкций, подвешенные к потолку помещений).
4) применение звукоизоляции (глушители).
5) применение средств индивидуальной защиты (наушники, шлемы, беру-ши).

Для снижения воздействия электромагнитного и ионизирующего излучения рекомендуется применять мониторы с пониженным уровнем излучения, устанавливать защитные экраны, а также соблюдать регламентированные режимы труда и отдыха.

Защита работника от негативного воздействия источника внешнего ионизирующего излучения достигается путем:

- снижение мощности источника излучения до минимально необходимой величины 
- увеличение расстояния между источником излучения и работником 
- уменьшение продолжительности работы в зоне излучения 
- установление между источником излучения и работником защитного

%\subsubsection{Мероприятия по защите от вредных производственных факторов}
%\subsubsection{Квалификационные требования к персоналу}

\subsection{Мероприятия по охране окружающей среды}

Для охраны окружающей среды необходимо разработать и освоить оптимальную технологию утилизации устаревших или пришедших в негодность внутренних заменяемых компонентов компьютера (интегральных схем, плат, микроконтроллеров, механических частей компьютера, шлейфов и~т.д.), а также внешних магнитных носителей.

Для этого на первом этапе утилизации необходимо сортировать и складировать в отдельные контейнеры отходы <<различной природы>> (отдельно провода, отдельно платы, отдельно различные механизмы, отдельно бумагу).

На втором этапе нужно отделять от неработающих деталей исправные части и использовать их в качестве запчастей для работающих изделий (если это возможно).

Оставшиеся --- сдавать в соответствующие профильные ремонтные или утилизирующие организации.

\subsection{Мероприятия по защите от чрезвычайных ситуаций}

В качестве основных направлений в решении задач обеспечения защиты от чрезвычайных ситуаций могут рассматриваться следующие:

1.	Прогнозирование и оценка возможных последствий чрезвычайных ситуа-ций.
2.	Планирование мероприятий по предотвращению или уменьшению вероятности возникновения чрезвычайных ситуаций, а также сокращению масштабов их последствий.
3.	Обеспечение устойчивой работы объектов народного хозяйства в чрезвы-чайных ситуациях.
4.	Обучение населения действиям в чрезвычайных ситуациях.
5.	Ликвидация последствий чрезвычайных ситуаций.

Для тушения пожаров в рассматриваемом помещении нужно использовать либо порошковые составы, либо установки углекислотного тушения, т.к. при использовании воды и пены велика вероятность поражения электрическим током.

Выбор типа огнетушителя (передвижной или ручной) обусловлен размерами возможных очагов пожара. При их значительных размерах необходимо использовать передвижные огнетушители.

Число огнетушителей одного из типов для разных категорий помещений необходимо устанавливать из таблиц, приведенных в нормах оснащения помещений ручными или передвижными огнетушителями.

Расстояние от возможного очага пожара до места размещения огнетушителя не должно превышать 30 м для помещений категории В и 70 м. для помещений категории Д.

Огнетушители должны всегда содержаться в исправном состоянии, периодически осматриваться, проверяться и своевременно перезаряжаться.

Для профилактики пожарной безопасности организуется обязательный инструктаж по правилам пожарной безопасности. Кроме этого необходимо наличие планов эвакуации и назначение ответственных лиц.

Рассматриваемое рабочее место оборудовано огнетушителем и системой пожарной сигнализации.

Дополнительных мер по защите от ЧС не требуется.

\subsection{Расчетная часть}

Work in process.

\subsection{Оценка эффективности принятых решений}

В данном разделе был произведен анализ основных вредных и опасных факторов исследуемого объекта.
По результатам анализа были разработаны мероприятия по обеспечению безопасных и комфортных условияй труда оператора ЭВМ.

Для проверки соответствия рабочий условий нормативным был произведен расчет освещенности.

Были разработаны мероприятия по охране окружающей среды и противостоянию возможным чрезвычайным ситуациям.

На основании выше изложенного, при условии выполнения всех мероприятий, соблюдения норм трудовой дисциплины и распорядка дня, рабочее место оператора персональной ЭВМ можно считать соответствующим классу труда 3.1.

Такие условия труда характеризуются такими отклонениями уровней вредных факторов от гигиенических нормативов, которые вызывают функциональные изменения, восстанавливающиеся, как правило, при более длительном (чем к началу следующей смены) прерывании контакта с вредными факторами и увеличивают риск повреждения здоровья.

\clearpage
\newpage