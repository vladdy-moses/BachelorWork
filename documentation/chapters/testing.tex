\section{Тестирование системы}

\subsection{Условия и порядок тестирования}

Объектом тестирования является портал подрядных организаций.

При каждом построении исполняемого модуля информационной системы происходит тестирование представлений на предмет ошибок времени исполнения.
Такой вид тестирования называется теневым, так как не предполагает каких-либо исходных данных.
Данный метод был реализован при помощи развёртывания сервера автоматического построения проектов TeamCity.

Нагрузочное тестирование предполагает собой проверку на максимальное число соединений или запросов от клиентов информационной системы, при котором последняя корректно функционирует.
Под корректным функционированием в данном случае принимается работа информационной системы без возникновения исключельных ситуаций при верно введённых данных.
Данное тестирование выполняется раз в месяц при помощи средств, представленных инструментальной средой разработки и платформой, на которой ведётся разработка информационной системы.

Для проверки графического интерфейса пользователя используются средства создания снимков веб-страниц.
К ним можно отнести такие программы как SlimerJS и PhantomJS.
Данные средства позволяют выполнять сценарии прохождения пользователя по сайту (симулируется работа реального пользователя).
На каждом шаге можно сделать снимок страницы и сохранить его в файловую систему.

\subsection{Исходные данные для контрольных примеров}

Work in process.

\subsection{Результаты тестирования}

Результаты теневого тестирования показали, что все представления, используемые в информационной системе, не содержат неверной разметки, что может повлиять на отображение информации в различных приложениях-браузерах.

Результаты нагрузочного тестировния показали, что система выдерживает до 40 запросов в секунду на сложные страницы (например, список конкурсов с фильтрацией по нескольким полям) и до 80 запросов в секунду на менее сложные страницы.
Система в состоянии обработать до 500 запросов в секунду на статическую информацию, например, страницу статистики или изображения.

Результаты тестирования графического интерфейса пользователя показали, что все страницы портала подрядчиков корректно отображаются на устройствах с диагональю экрана более 4,7 дюйма.
Удобное для пользования системой разрешение экрана -- 1366х768 и выше.

\clearpage
\newpage