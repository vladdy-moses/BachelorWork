\section{Техническое задание на создание системы}

\subsection{Назначение и цели создания системы}

Разрабатываемая информационная система должна быть предназначена для учёта заявок на выполнение подрядных работ по капитальному ремонту в субъекте федерации и проведению их розыгрыша, а также ведению отчётности по выполненным работам.

Основными целями создания системы являются:

\begin{easylist}
& увеличение числа заявок на подрядные работы по капитальному ремонту (КР);
& снижение издержек на участие в конкурсах на КР;
& увеличение прозрачности процесса отбора подрядных организаций на выполнение КР;
& улучшение механизмов взаимодействия подрядчиков с региональным оператором капитального ремонта (РОКР);
& раскрытие информации в электронном виде о заключённых договорах на капитальный ремонт;
& повышение производительности труда сотрудников подрядных организаций и РОКР.
\end{easylist}

\subsection{Характеристика объекта автоматизации}

Объектом автоматизации данной информационной системы является процесс работы с подрядными организациями при организации работ по капитальному ремонту.
Данный процесс охватывает подрядные организации в сфере строительства, а также регионального оператора капитального ремонта (примером такого оператора в Ульяновской области может служить <<Фонд модернизации ЖКХ>>).

\subsubsection{Общее описание}

Работа с подрядными организациями -- одина из ключевых обязанностей регионального оператора капитального ремонта наряду с утверждением программ развития жилого фонда и контролем качества проводимого капитального ремонта.
Именно от выбора подрядной организации зависит успех и долговечность ремонта.

\subsubsection{Структура и принципы функционирования}

Работа с подрядными организациями делится на несколько видов деятельности:

\begin{easylist}
& ведение учёта подрядных организаций;
& проведение конкурсов на капитальный ремонт;
& учёт плановых и фактических показателей проводимого капитального ремонта.
\end{easylist}

Основываясь на структуре, можно выделить основные функции объекта автоматизации:

\begin{easylist}
& формирование реестра подрядных организаций;
& выявление недобросовестных компаний;
& создание и розыгрыш конкурсов на проведение капитального ремонта;
& уведомление организации-победителя о решении РОКР;
& учёт и проверка плановых и фактических показателей проводимого капитального ремонта.
\end{easylist}

Все функции выполняются региональным оператором капитального ремонта при непосредственном участии подрядных организаций.

\subsubsection{Существующая информационная система и её недостатки}

На текущий момент практически все действия объекта автоматизации происходят в ручном режиме.
Исключение может составлять подсчёт победителя в конкурсе на проведение капитального ремонта, а также формирование писем подрядчикам.
Для этих целей используется офисный пакет Microsoft Office.

Недостатки такой информационной системы очевидны:

\begin{easylist}
& недостаточная прозрачность процесса одобрения организаций;
& отсутствие системности при учёте показателей капитального ремонта;
& несовершенные средства защиты информации;
& неоптимизированный процесс выдачи результатов одобрения организаций и итогов конкурсов.
\end{easylist}

Таким образом, указанные выше недостатки существующей информационной системы делают её недостаточно эффективной для использования в современных реалиях.

\subsubsection{Анализ аналогичных разработок}

В качестве аналогичных разработок можно взять торговые площадки, используемые для проведения конкурсов на проведение капитального ремонта различных субъектов.

Региональный оператор капитального ремонта города Санкт-Петербург использует Единую электронную торговую площадку, представленную в сети Интернет по адресу https://www.roseltorg.ru/.
Данный сайт полностью реализует требования Федерального закона N 44-ФЗ, в котором описывается механизм проведения торгов.
Также данная система поддерживает подпись контрактов через электронно-цифровую подпись.
Недостатком данной торговой площадки является низкая интеграция с жилищно-коммунальным хозяйством.
Данный недостаток очевиден ввиду ширины предметной области системы.
Ещё одним недостатком может являться отсутствие авторизации через единую систему идентификации и аутентификации (ЕСИА).
В дополнение к недостаткам в данной информационной системе можно отнести сложную регистрацию.

Региональным оператором капитального ремонта города Москвы для выбора подрядных организаций на проведение капитального ремонта используется портал подрядчиков города Москвы, доступный по адресу http://market.zakupki.mos.ru/.
На этом сайте также возможен вход при помощи электронной цифровой подписи.
Регистрация на портале простая, но требуется сертификат электронной подписи.
Недостатком данной системы является отсутствие интеграции с информационными системами жилищно-коммунального хозяйства.
Также недостатком является отсутствие авторизации через ЕСИА.

\subsubsection{Актуальность проводимой разработки}

Ввиду несовременности подходов к хранению и обработке информации, а также иных недостатков существующей информационной системы, требуется автоматизировать существующую ИС.
Автоматизированные системы сходного назначения не подходят полностью для данного объекта автоматизации из-за специфичности оного.

\subsection{Общие требования к системе}

\subsubsection{Требования к структуре и функционированию системы}

Разрабатываемая система должна состоять из двух частей: отдельной точки входа (типа веб-сайт) для подрядных организаций и модуля работы РОКР в ИС <<Объектовый учёт>>.
Данное разделение необходимо для поддержания взаимодействия регионального оператора капитального ремонта и подрядчиков без допуска последних до закрытой части ИС <<Объектовый учёт>>.

Также следует учитывать, что авторизоваться на веб-сайте имеют право также организации, уже имеющие доступ к ИС <<Объектовый учёт>> (например, управляющие компании).
Список типов организаций, имеющих право быть подрядчиком, определяется в ИС <<Объектовый учёт>>.
Авторизация для таких организаций должна быть единой.

В рамках модернизации системы можно рассмотреть возможность гибкой интеграции точки входа для подрядчиков с другими системами, реализующими \linebreak функциональность регионального сегмента ГИС ЖКХ.
Также следует предусмотреть возможность авторизации организаций через систему ЕСИА.

\subsubsection{Дополнительные требования}

В системе должен быть реализован механизм, который позволит защитить данные от несанкционированного доступа.

Система должна обрабатывать исключительные ситуации и корректно отображать сообщения об ошибках.

Должно быть предусмотрено ежедневное резервное копирование базы данных.

Персонал, работающий с информационной системой, должен обладать навыками работы за компьютером и использования интернет-обозревателя.

Система должна быть эргономичной.
Графический интерфейс пользователя должен отвечать современным требованиям к оформлению веб-сайтов, а также внутренним соглашениям по оформлению программных продуктов, принятым в ООО <<АИС Город>>.

\subsection{Требования к функциям, выполняемым системой}

\subsubsection{Учёт и отбор подрядных организаций}

В системе должна быть реализована процедура регистрации подрядной организации.
Регистрация может проходить как самостоятельно, так и региональным оператором капитального ремонта.

Для регистрации подрядчика необходимы следующие данные, представленные в табл.\ref{tab:tech-orgfields}.

\begin{footnotesize}
\begin{longtable}[h]{|p{0.5\textwidth}|p{0.2\textwidth}|p{0.2\textwidth}|}
	\caption{\label{tab:tech-orgfields}Необходимые поля подрядной организации} \\
	\hline
		\textbf{Название поля} & \textbf{Тип} & \textbf{Длина} \\
	\hline \endhead
		Полное название организации & текстовый & 500 \\
	\hline
		Краткое название организации & текстовый & 200 \\
	\hline
		Юридический адрес & текстовый & 500 \\
	\hline
		Физический адрес & текстовый & 500 \\
	\hline
		ИНН & текстовый & 20 \\
	\hline
		ОГРН & текстовый & 15 \\
	\hline
\end{longtable}
\end{footnotesize}

Дополнительно могут требоваться заполнение сведений, расположенных в табл.~\ref{tab:tech-orgaddfields}.

\begin{footnotesize}
\begin{longtable}[h]{|p{0.5\textwidth}|p{0.2\textwidth}|p{0.2\textwidth}|}
	\caption{\label{tab:tech-orgaddfields}Дополнительные поля подрядной организации} \\
	\hline
		\textbf{Название поля} & \textbf{Тип} & \textbf{Длина} \\
	\hline \endhead
		КПП & текстовый & 20 \\
	\hline
		Email организации & текстовый & 150 \\
	\hline
		Сайт организации в сети Интернет & текстовый & 150 \\
	\hline
		ФИО ответственного за участие в розыгрыше конкурсов лица & текстовый & 200 \\
	\hline
		Телефон ответственного за участие в розыгрыше конкурсов лица & текстовый & 50 \\
	\hline
		Численность профильных специалистов & числовой & 4 \\
	\hline
		Численность рабочих строителей & числовой & 4 \\
	\hline
		Наличие сертификата ISO 9001:2000 & логический & 1 \\
	\hline
		Наличие производственной базы & логический & 1 \\
	\hline
		Данные о наличии машин и механизмов & текстовый & MAX \\
	\hline
		Перечень заказчиков и адресов, на которых сейчас ведутся работы & текстовый & MAX \\
	\hline
\end{longtable}
\end{footnotesize}

РОКР может одобрять или отказывать в регистрации подрядчика на портале.
При отказе РОКР должен указать причину отказа.
Подрядчик вправе повторить заявку на его одобрение не чаще одного раза в сутки.

\subsubsection{Заполнение информации о сотрудниках подрядчика}

В информационной системе должен быть реализован механизм учёта сотрудников подрядной организации.

Список сведений о сотруднике представлен в табл.~\ref{tab:tech-employeefields}.

\begin{footnotesize}
\begin{longtable}[h]{|p{0.5\textwidth}|p{0.2\textwidth}|p{0.2\textwidth}|}
	\caption{\label{tab:tech-employeefields}Сведения о сотруднике} \\
	\hline
		\textbf{Название поля} & \textbf{Тип} & \textbf{Длина} \\
	\hline \endhead
		Фамилия & текстовый & 100 \\
	\hline
		Имя & текстовый & 100 \\
	\hline
		Отчество & текстовый & 100 \\
	\hline
		Должность & текстовый & 200 \\
	\hline
		Является руководителем & логический & 1 \\
	\hline
		Дата приёма на должность & дата & 6 \\
	\hline
		Дата освобождения от должности & дата & 6 \\
	\hline
\end{longtable}
\end{footnotesize}

Должность может быть выбрана из списка должностей ранее созданных записей сотрудников, а может быть создана новая.
При создании новой должности её название должно начинаться с заглавной буквы, в ней не должно быть двойных пробелов.
Подведение названия должности к данным требованием должно происходить автоматически.

В организации не может быть два руководителя на одну дату.

\subsubsection{Размещение конкурсов на проведение капитального ремонта}

В системе должна быть реализована возможность создания записей о конкурсах на капитальный ремонт.

Сведения о конкурсе представлены в табл.~\ref{tab:tech-contestfields}.

\begin{footnotesize}
\begin{longtable}[h]{|p{0.5\textwidth}|p{0.2\textwidth}|p{0.2\textwidth}|}
	\caption{\label{tab:tech-contestfields}Сведения о конкурсе} \\
	\hline
		\textbf{Название поля} & \textbf{Тип} & \textbf{Длина} \\
	\hline \endfirsthead
	\caption*{Продолжение таблицы \ref{tab:tech-contestfields}}\\
	\hline
		\textbf{Название поля} & \textbf{Тип} & \textbf{Длина} \\
	\hline \endhead
		Название & текстовый & 100 \\
	\hline
		Описание & текстовый & MAX \\
	\hline
		Дата публикации & дата & 6 \\
	\hline
		Дата начала приёма заявок & дата & 6 \\
	\hline
		Дата окончания приёма заявок & дата & 6 \\
	\hline
		Дата вскрытия конвертов & дата-время & 7 \\
	\hline
		Место вскрытия конвертов & текстовый & 200 \\
	\hline
		Является видимым подрядчикам & логический & 1 \\
	\hline
		Прикреплённый файл & бинарный & MAX \\
	\hline
		Является опубликованным в СМИ & логический & 1 \\
	\hline
		Название СМИ & текстовый & 100 \\
	\hline
		Номер периодики СМИ & текстовый & 10 \\
	\hline
		Число дней после закрытия конкурса, в течение которых должен быть заключен договор на капитальный ремонт & числовой & 4 \\
	\hline
		Дата внесения изменений & дата-время & 7 \\
	\hline
\end{longtable}
\end{footnotesize}

Конкурс может быть создан региональным оператором капитального ремонта.

Конкурс создаётся невидимым подрядчикам.
После заполнение информации хотя бы об одном лоте конкурс можно сделать видимым подрядчикам.

На конкурс может быть прикреплено множество лотов.
Сведения о лотах содержатся в табл.~\ref{tab:tech-lotfields}.

\begin{footnotesize}
\begin{longtable}[h]{|p{0.5\textwidth}|p{0.2\textwidth}|p{0.2\textwidth}|}
	\caption{\label{tab:tech-lotfields}Сведения о лоте} \\
	\hline
		\textbf{Название поля} & \textbf{Тип} & \textbf{Длина} \\
	\hline \endfirsthead
	\caption*{Продолжение таблицы \ref{tab:tech-lotfields}}\\
	\hline
		\textbf{Название поля} & \textbf{Тип} & \textbf{Длина} \\
	\hline \endhead
		Название & текстовый & 100 \\
	\hline
		Описание & текстовый & MAX \\
	\hline
		Причина нерозыгрыша & текстовый & 255 \\
	\hline
		Прикреплённый файл & бинарный & MAX \\
	\hline
		Является разыгранным & логический & 1 \\
	\hline
\end{longtable}
\end{footnotesize}

К каждому лоту может быть прикреплено множество смет.
Смета оформляется на конкретный объект жилого фонда и содержит список элементов капитального ремонта, которые необходимо отремонтировать в рамках данного конкурса.

Если лот со сметой не был разыгран, смета может быть использована повторно для описания лота другого конкурса.

Процесс размещения конкурса должен состоять из следующих шагов:

\begin{easylist}
& описание смет на капитальный ремонт;
& создание записи о конкурсе;
& создание записей о лотах конкурса;
& прикрепление смет к лотам;
& публикация конкурса подрядчикам.
\end{easylist}

\subsubsection{Ведение портфолио подрядчика}

Подрядчик должен иметь возможность вести портфолио.
В портфолио могут входить фотографии формата JPEG, PNG, GIF объёмом до 2 Мегабайт, а также документы формата PDF объёмом до 4 Мегабайт.

Портфолио одобренных подрядных организаций должно быть общедоступно.
Документы и фотографии должны быть разделены по блокам.

У записи в портфолио может быть описание.
Если описание отсутствует, у фотографий она принимает вид <<Запись №n>>, где n -- номер фотографии по порядку, начиная с 1.
У документов при отсутствии описания выводится имя файла.

РОКР имеет право просматривать портфолио любых подрядных организаций.

\subsubsection{Розыгрыш конкурсов на проведение капитального ремонта}

Подрядчик может подать заявку на участие в розыгрыше лота.
Данный лот должен являться частью действующего конкурса.
Действующим конкурсом считается конкурс, дата начала приёма заявок не позже сегодня, а дата окончания приёма заявок не раньше сегодня.

Список сведений в заявке представлен в табл.~\ref{tab:tech-bidfields}.

\begin{footnotesize}
\begin{longtable}[h]{|p{0.5\textwidth}|p{0.2\textwidth}|p{0.2\textwidth}|}
	\caption{\label{tab:tech-bidfields}Сведения о заявке на розыгрыш лота} \\
	\hline
		\textbf{Название поля} & \textbf{Тип} & \textbf{Длина} \\
	\hline \endfirsthead
	\caption*{Продолжение таблицы \ref{tab:tech-bidfields}}\\
	\hline
		\textbf{Название поля} & \textbf{Тип} & \textbf{Длина} \\
	\hline \endhead
		Комментарий подрядчика & текстовый & MAX \\
	\hline
		Дата подачи & дата-время & 7 \\
	\hline
		Прикреплённый файл & бинарный & MAX \\
	\hline
\end{longtable}
\end{footnotesize}

Информация об авторе заявки не должна быть видна другим участникам площадки и РОКР до даты вскрытия конвертов конкурса, на розыгрыш лота которого подавалась заявка.

% критерии отбора...
% Критерии отбора видны региональному оператору, когда наступает дата вскрытия конвертов.

% договор на кап.ремонт...
После наступления даты вскрытия конвертов в срок, указанный в поле <<Число дней после закрытия конкурса, в течение которых должен быть заключен договор на капитальный ремонт>> конкурса РОКР может заключить договора на проведения капитального ремонта по отдельным лотам.
Каждый лот конкурса может быть либо разыгран, либо не разыгран.
Если лот не разыгран, РОКР может указать причину нерозыгрыша.

Если лот конкурса разыгран, РОКР должен выбрать заявку-победителя, при этом создаётся проект договора на капитальный ремонт с автором заявки на розыгрыш.
Необходимо предусмотреть внесение в систему записи об успешном заключении договора между подрядчиком и региональным оператором капитального ремонта.
Также необходимо предоставить возможность вводить информацию об элементах договора, которые складываются из элементов смет, прикреплённых к разыгрываемому лоту.

\subsubsection{Размещение плановых показателей и отчётности по фактическим работам}

При заключении договора на капитальный ремонт подрядчик должен иметь возможность внести данные о плановых показателях по ремонту элементов договора.
В эти показатели входят сроки и планируемая сумма затрат на ремонт элемента договора.
Для подтверждения данных подрядчик может прикладывать документы, на основании которых были получены текущие плановые показатели.

При заключении договора на капитальный ремонт и одобрения плановых показателей, подрядчик должен иметь возможность размещать в системе отчётность по фактическим работам.
В системе должны быть указаны фактически сроки и затраты на выполнение каждого элемента договора на капитальный ремонт.
Также подрядчик должен иметь возможность прикреплять отсканированные копии смет КС-2 и КС-3.

РОКР может видеть записи о плановых показателях и фактических работах подрядчика.

Подрядчик не имеет право заполнять сведения о фактических работах только после одобрения региональным оператором капитального ремонта сведений о плановых показателях всех элементов договора на капитальный ремонт.

\subsubsection{Одобрение плановых показателей и отчётности по фактическим работам}

РОКР может одобрять записи о плановых показателях и фактических работах подрядчика, или отправить эти записи на доработку.

Подрядчик имеет право видеть только собственные сведения о плановых показателях и фактических работах.

\subsection{Требования к видам обеспечения}

\subsubsection{Требования к математическому обеспечению}

Необходимо разработать алгоритмы, реализующие следующие задачи:

\begin{easylist}
& отбора и сортировки конкурсов по различным полям;
& формирования запросов и обработка ответов от ЕСИА;
& вычисления хеш-значений пароля пользователя при помощи алгоритма \linebreak MD5 с использованием автогенерируемой соли;
& хранения учётных данных пользователя в рамках сессии;
& загрузи и проверки изображений форматов JPEG, PNG и GIF и документов формата PDF на наличие исполняемых компонентов.
\end{easylist}

При разработке информационной системы необходимо использовать возможности языка и платформы для реализации алгоритмов.
Если такая возможность отсутствует, необходимо использовать библиотеки, разрабатываемые по открытой лицензии.
При отсутствии такой возможности необходимо самостоятельно реализовывать необходимую функциональность.

\subsubsection{Требования к информационному обеспечению}

Для хранения данных необходимо использовать систему управления реляционными базами данных.

Требования к реляционной СУБД:

\begin{easylist}
& наличие сохраняемых запросов;
& наличие хранимых процедур;
& наличие триггеров на изменение и удаление записей в таблицах;
& наличие схем данных;
& расширенная обработка исключительных ситуаций;
& возможность использования транзакций;
& возможность создания резервных копий без отключения базы данных на техническое обслуживание.
\end{easylist}

Ввод данных в систему может быть как ручным, так и заноситься из других информационных систем.
Некоторые поля могут быть вычислены автоматически при использовании соответствующих алгоритмов.

В системе должен быть реализован многопользовательский режим доступа к данным.
Подключение системы к СУБД может быть однопользовательским.
При этом должна осуществляться проверка на доступ к данным в информационной системе.

В системе должно быть предусмотрено кеширование элементов экранных форм пользовательского интерфейса.
Время генерации страницы не должно превышать 5 секунд для экранных форм с закешированными элементами и не более 60 секунд для печатных форм.
Время загрузки страницы без закешированных элементов должно составлять не более 10 секунд.

Пользователям запрещается получать данные с помощью языка запросов без использования прикладного программного обеспечения.

При использовании системы необходимо обеспечить шифрование данных, используя протокол HTTPS.
Сертификат, используемый для подписи данных, не должен быть просрочен.
Запрещается использовать сертификаты, срок которых более 1 года.

Необходимо использовать резервное копирование базы данных раз в день.
При этом должны сохраняться резервные копии за предыдущие два дня, а также каждая копия, созданная в воскресенье, в течение последнего месяца.

\subsubsection{Требования к программному обеспечению}

Клиентская часть информационной системы должна корректно отображаться и функционировать на современных браузерах с долей использования на рынке более 3\% на трёх последних версиях.
В дополнение к вышеизложенным требованиям клиентская часть системы должна корректно функционировать в браузере Google Chrome версий моложе 10.

Экранные формы должны соответствовать правилам стилизации веб-ори\-ен\-ти\-ро\-ван\-ных информационных систем, принятым на момент разработки технического задания.

Для тестирования экранных форм могут применяться вспомогательные технологии, позволяющие делать снимок веб-страницы.
Примерами таких технологий являются PhantomJS, SlimerJS, Selenium.

Серверная часть информационной системы должна корректно разворачиваться на платформе Microsoft Windows Server 2008 R2,  с установленным .NET Framework версий 4 и 4.5.

Информационная система должна разрабатываться в IDE Microsoft Visual Studio или любых других открытых инструментальных средах разработки.

\subsubsection{Требования к техническому обеспечению}

Требования к аппаратному обеспечению клиентской части информационной системы:
\begin{easylist}
& процессор -- Intel Pentium 4 или аналог со схожими характеристиками;
& ОЗУ -- 512 Мб;
& НЖМД минимальным объёмом 10 Гб;
& видеоадаптер -- OpenGL-совместимый;
& монитор -- цветной с разрешением не менее 1024х768;
& сетевой адаптер.
\end{easylist}

Данные требования вызваны корректностью функционирования операционной системы Windows XP SP2 и браузера Google Chrome.

Требования к аппаратному обеспечению серверной части информационной системы:
\begin{easylist}
& процессор -- Intel Core i5 или аналог со схожими характеристиками;
& ОЗУ -- 16 Гб;
& НЖМД минимальным объёмом 1 Тб интерфейса SATA-III;
& видеоадаптер -- OpenCL-совместимый;
& сетевой адаптер с пропускной способностью 1 Гбит/с;
& источник бесперебойного питания выходной мощностью 2000 ВА.
\end{easylist}

Данные требования делают упор на скорость обработки информации для СУБД и хранение больших объёмов данных.
Также требования учитывают сетевую нагрузку и защиту от сбоев в электрооборудовании.

\clearpage
\newpage