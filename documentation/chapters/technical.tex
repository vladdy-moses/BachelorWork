\section{Техническое задание на создание системы}

\subsection{Назначение и цели создания системы}

Разрабатываемая информационная система должна быть предназначена для учёта заявок на выполнение подрядных работ по капитальному ремонту в субъекте федерации и проведению их розыгрыша, а также ведению отчётности по выполненным работам.

Основными целями создания системы являются:

\begin{enumerate}
	\item Увеличение числа заявок на подрядные работы по капитальному ремонту (КР);
	\item Снижение издержек на участие в конкурсах на КР;
	\item Увеличение прозрачности процесса отбора подрядных организаций на выполнение КР;
	\item Улучшение механизмов взаимодействия подрядчиков с региональным оператором капитального ремонта (РОКР);
	\item Раскрытие информации в электронном виде о заключённых договорах на капитальный ремонт;
	\item Повышение производительности труда сотрудников подрядных организаций и РОКР.
\end{enumerate}

\subsection{Характеристика объекта автоматизации}

Объектом автоматизации данной информационной системы является процесс работы с подрядными организациями при организации работ по капитальному ремонту.
Данный процесс охватывает подрядные организации в сфере строительства, а также регионального оператора капитального ремонта (примером такого оператора в Ульяновской области может служить <<Фонд модернизации ЖКХ>>).

\subsubsection{Общее описание}

Работа с подрядными организациями -- одина из ключевых обязанностей регионального оператора капитального ремонта наряду с утверждением программ развития жилого фонда и контролем качества проводимого капитального ремонта.
Именно от выбора подрядной организации зависит успех и долговечность ремонта.

\subsubsection{Структура и принципы функционирования}

Работа с подрядными организациями делится на несколько видов деятельности:

\begin{enumerate}
	\item Ведение учёта подрядных организаций;
	\item Проведение конкурсов на капитальный ремонт;
	\item Учёт плановых и фактических показателей проводимого капитального ремонта.
\end{enumerate}

Основываясь на структуре, можно выделить основные функции объекта автоматизации:

\begin{enumerate}
	\item Формирование реестра подрядных организаций;
	\item Выявление недобросовестных компаний;
	\item Создание и розыгрыш конкурсов на проведение капитального ремонта;
	\item Уведомление организации-победителя о решении РОКР;
	\item Учёт и проверка плановых и фактических показателей проводимого капитального ремонта.
\end{enumerate}

Все функции выполняются региональным оператором капитального ремонта при непосредственном участии подрядных организаций.

\subsubsection{Существующая информационная система и её недостатки}

На текущий момент практически все действия объекта автоматизации происходят в ручном режиме.
Исключение может составлять подсчёт победителя в конкурсе на проведение капитального ремонта, а также формирование писем подрядчикам.
Для этих целей используется офисный пакет Microsoft Office.

Недостатки такой информационной системы очевидны:

\begin{enumerate}
	\item недостаточная прозрачность процесса одобрения организаций;
	\item отсутствие системности при учёте показателей капитального ремонта;
	\item несовершенные средства защиты информации;
	\item неоптимизированный процесс выдачи результатов одобрения организаций и итогов конкурсов.
\end{enumerate}

Таким образом, указанные выше недостатки существующей информационной системы делают её недостаточно эффективной для использования в современных реалиях.

\subsubsection{Анализ аналогичных разработок}

В качестве аналогичных разработок можно взять торговые площадки, представленные в Интернете в широком количестве.

Work in process.

\subsubsection{Актуальность проводимой разработки}

Ввиду несовременности подходов к хранению и обработке информации, а также иных недостатков существующей информационной системы, требуется автоматизировать существующую ИС.
Автоматизированные системы сходного назначения не подходят полностью для данного объекта автоматизации из-за специфичности оного.

\subsection{Общие требования к системе}

\subsubsection{Требования к структуре и функционированию системы}

Разрабатываемая система должна состоять из двух частей: отдельной точки входа (типа веб-сайт) для подрядных организаций и модуля работы РОКР в ИС <<Объектовый учёт>>.
Данное разделение необходимо для поддержания взаимодействия регионального оператора капитального ремонта и подрядчиков без допуска последних до закрытой части ИС <<Объектовый учёт>>.

Также следует учитывать, что авторизоваться на веб-сайте имеют право также организации, уже имеющие доступ к ИС <<Объектовый учёт>> (например, управляющие компании).
Авторизация для таких организаций должна быть единой.

В рамках модернизации системы можно рассмотреть возможность гибкой игреграции точки входя для подрядчиков с другими системами, реализующими функциональность регионального секмента ГИС ЖКХ.
Также следует предусмотреть возможность авторизации организаций через систему ЕСИА, когда это будет необходимо.

\subsubsection{Дополнительные требования}

В системе должен быть реализован механизм, защищающий данные от несанкционированного доступа.

Система должна обрабатывать исключительные ситуации и корректно отображать сообщения об ошибках.

Должно быть предусмотрено ежедневное резервное копирование базы данных.

Персонал, работающий с информационной системой, должен обладать навыками работы за компьютером и использования интернет-обозревателя.

Система должна быть эргономичной.
Графический интерфейс пользователя должен отвечать современным требованиям к оформлению веб-сайтов, а также внутренним соглашениям по оформлению программных продуктов, принятым в ООО <<АИС Город>>.

\subsection{Требования к функциям, выполняемым системой}

\subsubsection{Название i-й функции}

Work in process.

\subsection{Требования к видам обеспечения}

\subsubsection{Требования к математическому обеспечению}

Work in process.

\subsubsection{Требования к информационному обеспечению}

Work in process.

\subsubsection{Требования к программному обеспечению}

Work in process.

\subsubsection{Требования к техническому обеспечению}

Work in process.

\clearpage
\newpage