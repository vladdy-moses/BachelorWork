\section{Техническое задание на создание системы}

\subsection{Назначение и цели создания системы}

Разрабатываемая информационная система должна быть предназначена для учёта заявок на выполнение подрядных работ по капитальному ремонту в субъекте федерации и проведению их розыгрыша, а также ведению отчётности по выполненным работам.

Основными целями создания системы являются:

\begin{enumerate}
	\item Увеличение числа заявок на подрядные работы по капитальному ремонту (КР);
	\item Снижение издержек на участие в конкурсах на КР;
	\item Увеличение прозрачности процесса отбора подрядных организаций на выполнение КР;
	\item Улучшение механизмов взаимодействия подрядчиков с региональным оператором капитального ремонта (РОКР);
	\item Раскрытие информации в электронном виде о заключённых договорах на капитальный ремонт;
	\item Повышение производительности труда сотрудников подрядных организаций и РОКР.
\end{enumerate}

\subsection{Характеристика объекта автоматизации}

Объектом автоматизации данной информационной системы является процесс работы с подрядными организациями при организации работ по капитальному ремонту.
Данный процесс охватывает подрядные организации в сфере строительства, а также регионального оператора капитального ремонта (примером такого оператора в Ульяновской области может служить <<Фонд модернизации ЖКХ>>).

\subsubsection{Общее описание}

Работа с подрядными организациями -- одина из ключевых обязанностей регионального оператора капитального ремонта наряду с утверждением программ развития жилого фонда и контролем качества проводимого капитального ремонта.
Именно от выбора подрядной организации зависит успех и долговечность ремонта.

\subsubsection{Структура и принципы функционирования}

Работа с подрядными организациями делится на несколько видов деятельности:

\begin{enumerate}
	\item Ведение учёта подрядных организаций;
	\item Проведение конкурсов на капитальный ремонт;
	\item Учёт плановых и фактических показателей проводимого капитального ремонта.
\end{enumerate}

Основываясь на структуре, можно выделить основные функции объекта автоматизации:

\begin{enumerate}
	\item Формирование реестра подрядных организаций;
	\item Выявление недобросовестных компаний;
	\item Создание и розыгрыш конкурсов на проведение капитального ремонта;
	\item Уведомление организации-победителя о решении РОКР;
	\item Учёт и проверка плановых и фактических показателей проводимого капитального ремонта.
\end{enumerate}

Все функции выполняются региональным оператором капитального ремонта при непосредственном участии подрядных организаций.

\subsubsection{Существующая информационная система и её недостатки}

На текущий момент практически все действия объекта автоматизации происходят в ручном режиме.
Исключение может составлять подсчёт победителя в конкурсе на проведение капитального ремонта, а также формирование писем подрядчикам.
Для этих целей используется офисный пакет Microsoft Office.

Недостатки такой информационной системы очевидны:

\begin{enumerate}
	\item недостаточная прозрачность процесса одобрения организаций;
	\item отсутствие системности при учёте показателей капитального ремонта;
	\item несовершенные средства защиты информации;
	\item неоптимизированный процесс выдачи результатов одобрения организаций и итогов конкурсов.
\end{enumerate}

Таким образом, указанные выше недостатки существующей информационной системы делают её недостаточно эффективной для использования в современных реалиях.

\subsubsection{Анализ аналогичных разработок}

В качестве аналогичных разработок можно взять торговые площадки, используемые для проведения конкурсов на проведение капитального ремонта различных субъектов.

Региональный оператор капитального ремонта города Санкт-Петербург использует Единую электронную торговую площадку, представленную в сети Интернет по адресу https://www.roseltorg.ru/.
Данный сайт полностью реализует требования Федерального закона N 44-ФЗ, в котором описывается механизм проведения торгов.
Также данная система поддерживает подпись контрактов через электронно-цифровую подпись.
Недостатком данной торговой площадки является низкая интеграция с жилищно-коммунальным хозяйством.
Данный недостаток очевиден ввиду ширины предметной области системы.
Ещё одним недостатком может являться отсутствие авторизации через единую систему идентификации и аутентификации (ЕСИА).
В дополнение к недостаткам в данной информационной системе можно отнести сложную регистрацию.

Региональным оператором капитального ремонта города Москвы для выбора подрядных организаций на проведение капитального ремонта используется портал подрядчиков города Москвы, доступный по адресу http://market.zakupki.mos.ru/.
На этом сайте также возможен вход при помощи электронной цифровой подписи.
Регистрация на портале простая, но требуется сертификат электронной подписи.
Недостатком данной системы является отсутствие интеграции с информационными системами жилищно-коммунального хозяйства.
Также недостатком является отсутствие авторизации через ЕСИА.

\subsubsection{Актуальность проводимой разработки}

Ввиду несовременности подходов к хранению и обработке информации, а также иных недостатков существующей информационной системы, требуется автоматизировать существующую ИС.
Автоматизированные системы сходного назначения не подходят полностью для данного объекта автоматизации из-за специфичности оного.

\subsection{Общие требования к системе}

\subsubsection{Требования к структуре и функционированию системы}

Разрабатываемая система должна состоять из двух частей: отдельной точки входа (типа веб-сайт) для подрядных организаций и модуля работы РОКР в ИС <<Объектовый учёт>>.
Данное разделение необходимо для поддержания взаимодействия регионального оператора капитального ремонта и подрядчиков без допуска последних до закрытой части ИС <<Объектовый учёт>>.

Также следует учитывать, что авторизоваться на веб-сайте имеют право также организации, уже имеющие доступ к ИС <<Объектовый учёт>> (например, управляющие компании).
Авторизация для таких организаций должна быть единой.

В рамках модернизации системы можно рассмотреть возможность гибкой игреграции точки входя для подрядчиков с другими системами, реализующими функциональность регионального секмента ГИС ЖКХ.
Также следует предусмотреть возможность авторизации организаций через систему ЕСИА, когда это будет необходимо.

\subsubsection{Дополнительные требования}

В системе должен быть реализован механизм, защищающий данные от несанкционированного доступа.

Система должна обрабатывать исключительные ситуации и корректно отображать сообщения об ошибках.

Должно быть предусмотрено ежедневное резервное копирование базы данных.

Персонал, работающий с информационной системой, должен обладать навыками работы за компьютером и использования интернет-обозревателя.

Система должна быть эргономичной.
Графический интерфейс пользователя должен отвечать современным требованиям к оформлению веб-сайтов, а также внутренним соглашениям по оформлению программных продуктов, принятым в ООО <<АИС Город>>.

\subsection{Требования к функциям, выполняемым системой}

\subsubsection{Учёт и отбор подрядных организаций}

В системе должна быть реализована процедура регистрации подрядной организации.
Регистрация может проходить как самостоятельно, так и региональным оператором капитального ремонта.

Для регистрации подрядчика необходимы следующие данные, представленные в таблице\ref{tab:tech-orgfields}.

\begin{footnotesize}
\begin{longtable}[h]{|p{0.5\textwidth}|p{0.2\textwidth}|p{0.2\textwidth}|}
	\caption{\label{tab:tech-orgfields}Необходимые поля подрядной организации} \\
	\hline
		\textbf{Название поля} & \textbf{Тип} & \textbf{Длина} \\
	\hline \endhead
		Полное название организации & текстовый & 40 \\
	\hline
		Краткое название организации & текстовый & 40 \\
	\hline
		Юридический адрес & текстовый & 40 \\
	\hline
		Физический адрес & текстовый & 40 \\
	\hline
		ИНН & текстовый & 40 \\
	\hline
		ОГРН & текстовый & 40 \\
	\hline
\end{longtable}
\end{footnotesize}

Дополнительно могут требоваться заполнение сведений, расположенных в таблице~\ref{tab:tech-orgaddfields}.

\begin{footnotesize}
\begin{longtable}[h]{|p{0.5\textwidth}|p{0.2\textwidth}|p{0.2\textwidth}|}
	\caption{\label{tab:tech-orgaddfields}Дополнительные поля подрядной организации} \\
	\hline
		\textbf{Название поля} & \textbf{Тип} & \textbf{Длина} \\
	\hline \endhead
		КПП & текстовый & 40 \\
	\hline
		Email организации & текстовый & 40 \\
	\hline
		Сайт организации в сети Интернет & текстовый & 40 \\
	\hline
		ФИО ответственного за участие в розыгрыше конкурсов лица & текстовый & 40 \\
	\hline
		Телефон ответственного за участие в розыгрыше конкурсов лица & текстовый & 40 \\
	\hline
		Численность профильных специалистов & текстовый & 40 \\
	\hline
		Численность рабочих строителей & текстовый & 40 \\
	\hline
		Наличие сертификата ISO 9001:2000 & логический & 1 \\
	\hline
		Наличие производственной базы & логический & 1 \\
	\hline
		Данные о наличии машин и механизмов & текстовый & 40 \\
	\hline
		Перечень заказчиков и адресов, на которых сейчас ведутся работы & текстовый & 40 \\
	\hline
\end{longtable}
\end{footnotesize}

РОКР может одобрять или отказывать в регистрации подрядчика на портале.
При отказе РОКР должен указать причину отказа.
Подрядчик вправе повторить заявку на его одобрение не чаще одного раза в сутки.

\subsubsection{Заполнение информации о сотрудниках подрядчика}

В информационной системе должен быть реализован механизм учёта сотрудников подрядной организации.

Список сведений о сотруднике представлен в таблице~\ref{tab:tech-employeefields}.

\begin{footnotesize}
\begin{longtable}[h]{|p{0.5\textwidth}|p{0.2\textwidth}|p{0.2\textwidth}|}
	\caption{\label{tab:tech-employeefields}Сведения о сотруднике} \\
	\hline
		\textbf{Название поля} & \textbf{Тип} & \textbf{Длина} \\
	\hline \endhead
		Фамилия & текстовый & 40 \\
	\hline
		Имя & текстовый & 40 \\
	\hline
		Отчество & текстовый & 40 \\
	\hline
		Должность & текстовый & 40 \\
	\hline
		Является руководителем & логический & 1 \\
	\hline
		Дата приёма на должность & дата & 6 \\
	\hline
		Дата освобождения от должности & дата & 6 \\
	\hline
\end{longtable}
\end{footnotesize}

Должность может быть выбрана из списка должностей ранее созданных записей сотрудников, а может быть создана новая.
При создании новой должности её название должно начинаться с заглавной буквы, в ней не должно быть двойных пробелов.
Подведение названия должности к данным требованием должно происходить автоматически.

В организации не может быть два руководителя на одну дату.

\subsubsection{Размещение конкурсов на проведение капитального ремонта}

В системе должна быть реализована возможность создания записей о конкурсах на капитальный ремонт.

Сведения о конкурсе представлены в таблице~\ref{tab:tech-contestfields}.

\begin{footnotesize}
\begin{longtable}[h]{|p{0.5\textwidth}|p{0.2\textwidth}|p{0.2\textwidth}|}
	\caption{\label{tab:tech-contestfields}Сведения о конкурсе} \\
	\hline
		\textbf{Название поля} & \textbf{Тип} & \textbf{Длина} \\
	\hline \endfirsthead
	\caption*{Продолжение таблицы \ref{tab:tech-contestfields}}\\
	\hline
		\textbf{Название поля} & \textbf{Тип} & \textbf{Длина} \\
	\hline \endhead
		Название & текстовый & 100 \\
	\hline
		Описание & текстовый & MAX \\
	\hline
		Дата публикации & дата & 6 \\
	\hline
		Дата начала приёма заявок & дата & 6 \\
	\hline
		Дата окончания приёма заявок & дата & 6 \\
	\hline
		Дата вскрытия конвертов & дата-время & 7 \\
	\hline
		Место вскрытия конвертов & текстовый & 200 \\
	\hline
		Является видимым подрядчикам & логический & 1 \\
	\hline
		Прикреплённый файл & бинарный & MAX \\
	\hline
		Является опубликованным в СМИ & логический & 1 \\
	\hline
		Название СМИ & текстовый & 100 \\
	\hline
		Номер периодики СМИ & текстовый & 10 \\
	\hline
		Число дней после закрытия конкурса, в течение которых должен быть заключен договор на капитальный ремонт & числовой & 4 \\
	\hline
		Дата внесения изменений & дата-время & 7 \\
	\hline
\end{longtable}
\end{footnotesize}

Конкурс может быть создан региональным оператором капитального ремонта.

Конкурс создаётся невидимым подрядчикам.
После заполнение информации хотя бы об одном лоте конкурс можно сделать видимым подрядчикам.

На конкурс может быть прекрплено множество лотов.
Сведения о лотах содержатся в таблице~\ref{tab:tech-lotfields}.

\begin{footnotesize}
\begin{longtable}[h]{|p{0.5\textwidth}|p{0.2\textwidth}|p{0.2\textwidth}|}
	\caption{\label{tab:tech-lotfields}Сведения о лоте} \\
	\hline
		\textbf{Название поля} & \textbf{Тип} & \textbf{Длина} \\
	\hline \endfirsthead
	\caption*{Продолжение таблицы \ref{tab:tech-lotfields}}\\
	\hline
		\textbf{Название поля} & \textbf{Тип} & \textbf{Длина} \\
	\hline \endhead
		Название & текстовый & 100 \\
	\hline
		Описание & текстовый & MAX \\
	\hline
		Причина нерозыгрыша & текстовый & 255 \\
	\hline
		Прикреплённый файл & бинарный & MAX \\
	\hline
		Является разыгранным & логический & 1 \\
	\hline
\end{longtable}
\end{footnotesize}

\subsubsection{Розыгрыш конкурсов на проведение капитального ремонта}

Work in process.

\subsubsection{Размещение плановых показателей и отчётности по фактическим работам}

Work in process.

\subsubsection{Одобрение плановых показателей и отчётности по фактическим работам}

Work in process.

\subsection{Требования к видам обеспечения}

\subsubsection{Требования к математическому обеспечению}

Work in process.

\subsubsection{Требования к информационному обеспечению}

Work in process.

\subsubsection{Требования к программному обеспечению}

Work in process.

\subsubsection{Требования к техническому обеспечению}

Work in process.

\clearpage
\newpage