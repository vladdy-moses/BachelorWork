\section{Экономический раздел}

\subsection{Оценка трудоёмкости}

Трудоёмкость --- это  показатель, характеризующий затраты рабочего времени на производство определённой потребительной стоимости или на выполнение конкретной технологической операции\cite{marks}.
Трудоёмкость определяет эффективность использования одного из главных производственных ресурсов — рабочей силы. На величину трудоёмкости влияет ряд факторов: технический уровень производства (фондовооружённость труда и энерговооружённость труда, полезные свойства предметов труда, технология), квалификация работников, организация и условия труда, сложность изготовляемой продукции и др.
В узком смысле под трудоёмкостью понимаются средние затраты живого труда на единицу или на весь объём изготовленной продукции. Мера измерения --- рабочее время.
Показатель трудоемкости является обратным показателю производительности труда и рассчитывается по формуле \ref{eq:trudoemk}:

\begin{equation}
	\label{eq:trudoemk}
	T = \frac{T_{w}}{C_{p}},
\end{equation}
\begin{ESKDexplanation}
	\item где $T$ --- трудоемкость;
	\item $T_{w}$ --- рабочее время;
	\item $C_{p}$ --- количество произведенной продукции.
\end{ESKDexplanation}

При оценке трудоемкости разработки ИС следует учитывать особенности данного вида продукции.
Как правило на выходе получается один законченный продукт, на разработку которого потрачены все усилия.

Трудоемкость разработки системы прямо зависит от того, сколько времени занимает каждый этап разработки системы. Подходить к прогнозированию того, сколько времени займет тот или иной этап разработки, нужно крайне ответственно, чтобы свести погрешности в оценке трудоемкости работ к минимальным значениям.

В настоящее время для оценки трудоемкости разработки ИС применяется метод оценки работ в человеко-часах. Этот метод показал свою эффективность как при разработке автоматизированной системы одним человеком, так и при разработке одной системы целой командой разработчиков.

Для определения общей трудоемкости разработки системы целесообразно рассмотреть трудоемкости каждого этапа разработки системы в отдельности, после чего произвести расчет общей трудоемкости по формуле \ref{eq:trudoemk_sum}:

\begin{equation}
	\label{eq:trudoemk_sum}
	T_{gen} = \sum^{n}_{i=1}t_{i},
\end{equation}
\begin{ESKDexplanation}
	\item где $T_{gen}$ --- общая трудоемкость разработки системы;
	\item $t_{i}$ --- трудоемкость работ на i-й стадии разработки;
	\item $n$ --- количество стадий.
\end{ESKDexplanation}

Основываясь на ГОСТ 34.601-90\cite{gost34601} можно выделить следующие стадии и этапы разработки информационной системы, а также оценки трудоёмкости по каждой стадии, представленной в таблице \ref{tab:trudoemk}.

\begin{footnotesize}
\begin{longtable}[h]{|p{0.2\textwidth}|p{0.5\textwidth}|p{0.2\textwidth}|}
	\caption{\label{tab:trudoemk}Распределение времени разработки системы по стадиям и видам работ с оценкой их трудоемкости} \\
	\hline
		\textbf{Стадии} &
		\textbf{Этапы работ} &
		\textbf{Трудоёмкость, чел*ч} \\
	\hline
		Формирование требований &
		Обследование объекта и обоснование необходимости создания АС. \par Формирование требований пользователя к АС. &
		40 \\
	\hline
		Разработка концепции АС &
		Изучение объекта. \par Проведение необходимых научно-исследовательских работ. \par Разработка вариантов концепции АС, удовлетворяющего требованиям пользователя. &
		60 \\
	\hline
		Техническое задание &
		Разработка и утверждение технического задания на создание АС. &
		40 \\
	\hline
		Эскизный проект &
		Разработка предварительных проектных решений по системе и её частям. \par Разработка документации на АС и её части. &
		120 \\
	\hline
		Технический проект &
		Разработка проектных решений по системе и её частям. \par Разработка документации на АС и её части. \par Разработка и оформление документации на поставку изделий для комплектования АС и (или) технических требований (технических заданий) на их разработку. &
		120 \\
	\hline
		Рабочая документация &
		Разработка рабочей документации на систему и её части. &
		32 \\
	\hline
		Тестирование &
		Проведение предварительного тестирования. \par Проведение опытной эксплуатации. \par Проведение приемочного тестирования. &
		80 \\
	\hline
		Ввод в действие &
		Подготовка объекта автоматизации к вводу АС в действие. \par Подготовка персонала. \par Комплектация АС поставляемыми изделиями (программными и техническими средствами, программно-техническими комплексами, информационными изделиями). \par Пусконаладочные работы. &
		24 \\
	\hline
\end{longtable}
\end{footnotesize}

Таким образом, общая трудоемкость разработки системы составила 516 часов, что равно 64,5 восьмичасовым рабочим дням.

\subsection{Расчёт затрат}

\subsubsection{Расчёт затрат на материальные ресурсы}

К материальным ресурсам относятся различные виды сырья, материалов, топлива, энергии, комплектующих и полуфабрикатов, которые организации приобретают для использования в своей хозяйственной деятельности с целью выпуска продукции, выполнения работ, оказания услуг.

Совокупность затрат на приобретение материальных ресурсов называется материальными затратами, являющимися одним из экономических элементов стоимости готовой продукции.
Следовательно, чем меньше материальные затраты, тем меньше себестоимость, что позволяет предприятию увеличить прибыль от реализации продукции.

Что касается разработки ИС, то в данной сфере материальные затраты не столь значительны.
И основная доля затрат приходится на оплату труда разработчиков.

Расчет затрат на материальные ресурсы производится по формуле \ref{eq:zatrat}:

\begin{equation}
	\label{eq:zatrat}
	C_{mat} = \sum^{n}_{i=1} E_{i} \times C_{i},
\end{equation}
\begin{ESKDexplanation}
	\item где $C_{mat}$ --- затраты на материальные ресурсы;
	\item $E_{i}$ --- расход i-го вида материального ресурса, измеренный в натуральных величинах;
	\item $C_{i}$ ---  цена за единицу i-го вида материального ресурса, измеренная в руб.;
	\item $i$ --- вид материального ресурса;
	\item $n$ --- общее количество используемых видов материальных ресурсов.
\end{ESKDexplanation}

Ресурсы, используемые при разработке программного продукта представлены в табл. \ref{tab:zatrat}. 

\begin{footnotesize}
\begin{longtable}[h]{|p{0.05\textwidth}|p{0.25\textwidth}|p{0.1\textwidth}|p{0.15\textwidth}|p{0.15\textwidth}|p{0.15\textwidth}|}
	\caption{\label{tab:zatrat}Затраты на материальные ресурсы} \\
	\hline
		\textbf{№} &
		\textbf{Наименование} &
		\textbf{Единица измерения} &
		\textbf{Требуемое количество единиц} &
		\textbf{Цена за единицу товара, руб} &
		\textbf{Сумма, руб} \\
	\hline
		1 & Ноутбук & шт & 1 & 25000 & 25000 \\ \hline
		2 & Мышь компьютерная & шт & 1 & 500 & 500 \\ \hline
		3 & Клавиатура & шт & 1 & 350 & 350 \\ \hline
		4 & Бумага писчая & упак & 1 & 250 & 250 \\ \hline
		5 & Канцелярские товары & шт & 1 & 500 & 500 \\ \hline
		6 & USB-флеш накопитель & шт & 1 & 800 & 800 \\ \hline
		\multicolumn{5}{|l|}{Итоговая сумма:} & 27400 \\ \hline
\end{longtable}
\end{footnotesize}

По итогам расчета затраты на материальные ресурсы, требуемые при разработке ИС, составили 27400 рублей.

Также необходимо произвести расчет стоимости расходных материалов, требуемых для разработки данной ИС. Затраты на расходные материалы представлены в табл. \ref{tab:zatrat_rash}.

\begin{footnotesize}
\begin{longtable}[h]{|p{0.05\textwidth}|p{0.25\textwidth}|p{0.1\textwidth}|p{0.15\textwidth}|p{0.15\textwidth}|p{0.15\textwidth}|}
	\caption{\label{tab:zatrat_rash}Затраты на расходные материалы} \\
	\hline
		\textbf{№} &
		\textbf{Наименование} &
		\textbf{Единица измерения} &
		\textbf{Требуемое количество единиц} &
		\textbf{Цена за единицу товара, руб} &
		\textbf{Сумма, руб} \\
	\hline
		1 & Оплата услуг интернет-провайдера & руб/мес & 3 & 650 & 1950 \\ \hline
		\multicolumn{5}{|l|}{Итоговая сумма:} & 1950 \\ \hline
\end{longtable}
\end{footnotesize}

По итогам расчета сумма затрат на расходные материалы, требуемые при разработке ИС, составили 1950 рублей. 

\subsubsection{Расчёт затрат на электроэнергию}

Расчет стоимости электроэнергии, затрачиваемой на разработку системы, производится на основе действующих тарифов на электроэнергию, устанавливаемых региональными энергетическими комиссиями.

Общая сумма затрат на электроэнергию рассчитывается по формуле \ref{eq:power}.

\begin{equation}
	\label{eq:power}
	C_{power} =  \sum^{n}_{i=1}M_i \times T_i \times R,
\end{equation}
\begin{ESKDexplanation}
	\item где $C_{power}$ --- сумма затрат на электроэнергию;
	\item $M_{i}$ --- паспортная мощность i-го электрооборудования, кВт;
	\item $T_{i}$ --- время работы i-го оборудования за весь период разработки, ч;
	\item $R$ --- тариф электроэнергии, руб./кВт*ч;
	\item $i$ --- вид электрооборудования;
	\item $n$ --- количество электрооборудования.
\end{ESKDexplanation}

Суммарные затраты на электроэнергию представлены в табл. \ref{tab:zatrat_power}.

\begin{footnotesize}
\begin{longtable}[h]{|p{0.05\textwidth}|p{0.2\textwidth}|p{0.2\textwidth}|p{0.15\textwidth}|p{0.15\textwidth}|p{0.1\textwidth}|}
	\caption{\label{tab:zatrat_power}Затраты на электроэнергию} \\
	\hline
		\textbf{№} &
		\textbf{Наименование} &
		\textbf{Паспортная мощность, кВт} &
		\textbf{Суммарное время работы оборудования за время разработки, ч} &
		\textbf{Тариф на электроэнергию, руб./кВт*ч} &
		\textbf{Сумма, руб} \\
	\hline
		1 & Ноутбук & 0,06 & 516 & 2,07 & 64,09 \\ \hline
		2 & Искусственное освещение & 0,08 & 516 & 2,07 & 85,45 \\ \hline
		\multicolumn{5}{|l|}{Итоговая сумма:} & 149,54 \\ \hline
\end{longtable}
\end{footnotesize}

По итогам расчета общие затраты на электроэнергию, требуемые на разработку ИС, составили 149 рублей 54 копейки.

Так как разработка будет проводиться весной, то в расчете стоимости отопления нет необходимости.

\subsubsection{Расчёт заработной платы с начислениями}

Зарплата начисляется, исходя из установленных на предприятии тарифов, сдельных расценок, окладов и сведений о фактически отработанном работниками времени или сведений об объемах выпущенной продукции.
Расчет зарплаты производится на основании таких документов, как штатное расписание, положение об оплате труда, приказы о приеме на работу и трудовые договоры.

Данными документами устанавливается размер и форма оплаты труда конкретного работника.
Кроме того, существуют документы, на основании которых зарплата может быть изменена в большую или меньшую сторону: служебные записки, приказы о премировании и т.д.
Положение об оплате труда предусматривает поощрительные выплаты и порядок начисления зарплаты применительно к каждой категории работников предприятия. 

Существует несколько форм оплаты труда: повременная и сдельная. При повременной оплате труда зарплата выплачивается в зависимости от отработанного времени и от квалификации работника.
При сдельной оплате труда зарплата зависит от количества произведенной продукции. 

При сдельной оплате труда расчет производится исходя из сдельных расценок, установленных на изделие, и количества обработанных изделий. Чаще всего сдельные расценки, установленные на изготовление единицы продукции, постоянны, поэтому заработок рабочего можно определить как произведение сдельной расценки на объем изготовленной продукции.

При повременной форме оплаты труда зарплата зависит от количества отработанного времени. Для учета фактически отработанного времени ведется табель учета рабочего времени и табель расчета заработной платы.

Для расчета суммы основной заработной платы разработчика системы следует использовать формулу \ref{eq:zarplata}. 

\begin{equation}
	\label{eq:zarplata}
	Sal =  \sum^{n}_{i=1} R_i \times T_i,
\end{equation}
\begin{ESKDexplanation}
	\item где $Sal$ --- сумма основной заработной платы разработчика системы;
	\item $R_i$ --- часовая ставка i-го работника, руб.;
	\item $T_{i}$ --- время на разработку системы, ч;
	\item $i$ --- порядковый номер работника;
	\item $n$ --- количество работников.
\end{ESKDexplanation}

Суммарные затраты на основную оплату труда разработчика представлены в табл. \ref{tab:zarplata}.

\begin{footnotesize}
\begin{longtable}[h]{|p{0.05\textwidth}|p{0.2\textwidth}|p{0.25\textwidth}|p{0.2\textwidth}|p{0.15\textwidth}|}
	\caption{\label{tab:zarplata}Затраты на основную оплату труда} \\
	\hline
		\textbf{№} &
		\textbf{Категория работника} &
		\textbf{Трудоемкость разработки, чел*ч.} &
		\textbf{Часовая ставка, руб./ч.} &
		\textbf{Сумма, руб.} \\
	\hline
		1 & Разработчик системы & 516 & 150 & 82560 \\ \hline
		\multicolumn{4}{|l|}{Итоговая сумма:} & 82560 \\ \hline
\end{longtable}
\end{footnotesize}

Дополнительная заработная плата составляет 0\% от основной заработной платы и составляет премию сотрудника.
Она учитывается так же, как и основная, и включается в фонд заработной платы.
Суммарная заработная плата $СЗ_{тр}$ рассчитывается по формуле \ref{eq:zarplata_dop}: 

\begin{equation}
	\label{eq:zarplata_dop}
	Sal_{add} = Sal + 0\% = 82560~\text{руб}.
\end{equation}

Таким образом суммарная заработная плата разработчика за время разработки системы составит 82560 руб. 

После того как определена заработная плата сотрудника необходимо произвести расчет налоговых отчислений, которые предприятие обязано выплатить в пользу государства за сотрудника.
Руководствоваться следует Налоговым Кодексом Российской Федерации.

Результаты расчета обязательных взносов для работника, родившегося позже 1967 года, приведены в табл. \ref{tab:zarplata_nalog}.

\begin{table}[h!]
\caption{\label{tab:zarplata_nalog}Отчисления на обязательные взносы}
\begin{footnotesize}
\begin{tabular}{|l|c|}
	\hline
	\textbf{Вид отчисления} & \textbf{Размер, \%} \\
	\hline 
	Пенсионный фонд -- страховая часть & 16 \\ 
	Пенсионный фонд -- накопительная часть & 6 \\
	ФФОМС & 5,1 \\ 
	ФСС & 2,9 \\ 
	Страхование от несчастных случаев & 0,2 \\
	\hline
	\textbf{Налоговая нагрузка} & 30,2 \\ \hline
	\textbf{Начисленно заработной платы, руб.} & 82560 \\ \hline
	\textbf{Сумма взносов, руб.} & 24933,12 \\ \hline
\end{tabular}
\end{footnotesize}
\end{table}

\subsubsection{Расчёт амортизационных отчислений}

Расходы на амортизацию можно рассчитать по формуле \ref{eq:amortiz}:

\begin{equation}
	\label{eq:amortiz}
	C_{a} = \sum^{n}_{i=1} \frac{C_i \times N_{ai} \times T_{wi}}{100 \times T_{ei}},
\end{equation}
\begin{ESKDexplanation}
	\item где $C_{a}$ --- расходы на амортизацию;
	\item $C_i$ --- стоимость i-го оборудования, руб.;
	\item $N_{ai}$ --- годовая норма амортизации i-го оборудования, \%;
	\item $T_{wi}$ --- время работы i-го оборудования за весь период разработки, ч;
	\item $T_{ei}$ --- эффективный фонд времени работы i-го оборудования за год, ч/год;
	\item $i$ --- вид оборудования;
	\item $n$ --- количество оборудования.
\end{ESKDexplanation}

При норме амортизации равной 20\% расходы на амортизацию составят:
\begin{equation*}
\begin{split}
	C_{a} = 
	\frac{25000 \times 20 \times 516}{100 \times 1986} + 
	\frac{500 \times 20 \times 516}{100 \times 1986} + \\
	\frac{350 \times 20 \times 516}{100 \times 1986} + 
	\frac{800 \times 20 \times 516}{100 \times 1986} =
	1384,83~\text{руб}.
\end{split}
\end{equation*}

\subsection{Расчёт себестоимости разработки}

Себестоимость разработки системы определяется совокупностью стоимостей, материальных ресурсов проекта, расходных материалов, стоимости электрической энергии и оплаты труда разработчику.

Итоговый расчет себестоимости проекта приведен в табл. \ref{tab:sebest}.

\begin{footnotesize}
\begin{longtable}[h]{|p{0.05\textwidth}|p{0.65\textwidth}|p{0.2\textwidth}|}
	\caption{\label{tab:sebest}Себестоимость проекта} \\
	\hline
		\textbf{№} &
		\textbf{Статья затрат} &
		\textbf{Сумма, руб.} \\
	\hline
		1 & Затраты на материальные ресурсы & 27400 \\ \hline
		2 & Затраты на расходные материалы & 1950 \\ \hline
		3 & Затраты на электроэнергию & 149,54 \\ \hline
		4 & Затраты на оплату труда разработчика & 82560 \\ \hline
		5 & Затраты на обязательные взносы & 24933,12 \\ \hline
		6 & Затраты на амортизацию основных фондов & 1384,83 \\ \hline
		\multicolumn{2}{|l|}{Итого:} & 138377,49 \\ \hline
\end{longtable}
\end{footnotesize}

\subsection{Расчёт плановой прибыли}

После расчёта совокупных затрат, требуемых для разработки ИС, необходимо рассчитать прибыль, которую получит предприятие от данной разработки, так как любая разработка выполняется исключительно с целью повышения эффективности какого-либо вида деятельности, сокращения временных и материальных затрат производства.

Информация о затратах и прибылях по типичным услугам, связанным с работой системы, приведены в табл.~\ref{tab:ecomonics_services}.

\begin{footnotesize}
\begin{longtable}[h]{|p{0.3\textwidth}|p{0.1\textwidth}|p{0.05\textwidth}|p{0.1\textwidth}|p{0.15\textwidth}|p{0.1\textwidth}|}
	\caption{\label{tab:ecomonics_services}Список затрат и прибылей, связанных с поддержкой ИС в месяц} \\
	\hline
		\textbf{Услуга} &
		\textbf{Частота в мес.} &
		\textbf{Цена} &
		\textbf{Часов до внедрения} &
		\textbf{Часов после внедрения} &
		\textbf{Прибыль} \\
	\hline
		Регистрация подрядчика & 10 & 100 & 0,5 & 0 & 500,00 \\ \hline
		Рассылка информации о конкурсе & 15 & 100 & 2 & 0,5 & 2250,00 \\ \hline
		Формирование лотов & 100 & 150 & 1 & 0,25 & 11250,00 \\ \hline
		Проведение конкурса & 15 & 150 & 2 & 0,5 & 3375,00 \\ \hline
		Проверка подрядчика & 45 & 100 & 1 & 0,25 & 3375,00 \\ \hline
		Подача заявки & 100 & 100 & 2 & 0,5 & 15000,00 \\ \hline
		Регистрация у РОКР & 10 & 100 & 3 & 0,5 & 2500,00 \\ \hline
		Отчётность по работам & 50 & 150 & 6 & 2 & 30000,00 \\ \hline
		Информирование по работам & 50 & 100 & 1 & 0,25 & 3750,00 \\ \hline
		Поддержка ИС & 3 & 160 & 0 & 80 & -38400,00 \\ \hline
		Затраты на сервера & 2 & 10 & 0 & 744 & -14880,00 \\ \hline
		\multicolumn{5}{|l|}{Итого:} & 18720,00 \\ \hline
\end{longtable}
\end{footnotesize}

Согласно информации, приведённой выше, прибыль от разработки информационной системы составляет 18720,00 руб.

Число месяцев, за которые система окупится, составляет:
\begin{equation*}
\begin{split}
	O = [\frac{138377,49}{18720,00}] = 8~\text{мес}.
\end{split}
\end{equation*}

Рентабельность разработки считается как обратная величина от окумаемости на время разработки и составляет 13,53~\%.
Данная величина является приемлемой.

\subsection{Выводы по технико-экономическому анализу }

В ходе проведенного технико-экономического анализа разработки системы была рассчитана себестоимость разработки системы, которая составила 138377,49 руб.

Прибыль в месяц от данной разработки за вычетом налога составит 18720,00 руб.
Программный продукт окупается за 8 месяцев.

Таким образом можно сделать вывод о том, что, потратив деньги на разработку данной системы, заинтересованные организации в лице региональных операторов капитального ремонта в конечном итоге смогут сократить последующие затраты на выполнение процессов, автоматизированных разрабатываемой системой.
Для компании ООО <<ИнтелСофт>> затраты на разработку системы полностью окупятся, и она начнёт приносить прибыль ввиду использования модели распространения программного обеспечения "software as a service".

%\subsubsection{ заголовок}
%\point
%\subpoint
%\subsubpoint

\clearpage
\newpage
