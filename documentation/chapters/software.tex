\section{Программное обеспечение системы}

% \subsection{Структура программного обеспечения и функции его компонентов}
% Work in process.

\subsection{Выбор компонентов программного обеспечения}

\subsubsection{Инструментальное средство разработки и язык программирования}

На данный момент существует огромное множество языков программирования, на которых можно разработать веб-ориентированную информационную систему.

Для выбора необходимого языка программирования была составлена сравнительная характеристика трёх самых распространённых и динамично развивающихся языков, используемых в сети Интернет~\cite{chikagosHub,leonardTeo}.
Результаты анализа приведены в таблице~\ref{tab:software-language}.

\begin{footnotesize}
\begin{longtable}[h]{|p{0.3\textwidth}|p{0.2\textwidth}|p{0.2\textwidth}|p{0.2\textwidth}|}
	\caption{\label{tab:software-language}Сравнительных анализ языков программирования веб-ориентированных ИС} \\
	\hline
		\textbf{Критерий} &
		\textbf{PHP} &
		\textbf{Ruby} &
		\textbf{C\#} \\
	\hline \endfirsthead
	\hline
		\textbf{Критерий} &
		\textbf{PHP} &
		\textbf{Ruby} &
		\textbf{C\#} \\
	\hline \endhead
	Скорость формирования страницы & 
	средне & медленно & быстро \\ \hline
	
	Нагрузка пользователей & 
	мало & мало & много \\ \hline
	
	Лёгкость и скорость разработки & 
	легко & очень легко & средне \\ \hline
	
	Наличие фреймворков & 
	есть & есть & есть \\ \hline
	
	Расширенная обработка исключений & 
	нет	& есть & есть \\ \hline
	
	Сложность генерации отчётов & 
	высокая & высокая & низкая \\ \hline
\end{longtable}
\end{footnotesize}

Сложность генерации отчётов включает сложность создания и отображения отчёта, предоставляемого пользователю.

Ввиду того, что информационная система будет открытой для гостей, необходимо иметь высокую скорость формирования динамических страниц.
Также в техническом задании указано наличие расширенной обработки исключений и отклика отчётных форм.
На основании данных критериев, а также используемого языка при написании регионального сегмента ГИС ЖКХ, был выбран язык программирования C\# и, соответственно, платформа .NET.

Для данного языка программирования существует несколько программных средств: разрабатываемая корпорацией Microsoft среда разработки Visual Studio, свободная среда разработки Xamarin Studio, а также другие менее функциональные аналоги.
Сравнительный анализ Visual Studio и Xamarin Studio представлен в таблице~\ref{tab:software-sharpide}.

\begin{footnotesize}
\begin{longtable}[h]{|p{0.5\textwidth}|p{0.2\textwidth}|p{0.2\textwidth}|}
	\caption{\label{tab:software-sharpide}Сравнительный анализ средств разработки на C\#} \\
	\hline
		\textbf{Критерий} &
		\textbf{Visual Studio} &
		\textbf{Xamarin} \\
	\hline \endfirsthead
	\hline
		\textbf{Критерий} &
		\textbf{Visual Studio} &
		\textbf{Xamarin} \\
	\hline \endhead
	Подсветка синтаксиса & 
	есть & есть \\ \hline
	
	Дизайнер веб-форм & 
	есть & нет \\ \hline
	
	Дизайнер модели данных & 
	есть & нет \\ \hline
	
	Кроссплатформенность & 
	нет & есть \\ \hline
	
	Интеграция с CVS & 
	есть & есть \\ \hline
\end{longtable}
\end{footnotesize}

Как можно видеть из представленного анализа, оба инструментальных средства имеют преимущества друг перед другом.
Но так как дизайнер модели данных и веб-форм важнее кроссплатформенности (для разработки можно установить и виртуальную машину с необходимой ОС), то для разработки было принято решение использовать Microsoft Visual Studio.

Согласно спецификации Microsoft, для поддержки .NET framework 4 и 4.5 (требуется техническим заданием) необходима Microsoft Visual Studio, начиная с версии 2012~\cite{vs_mdn}.
Была выбрана версия 2013 ввиду увеличения удобства пользования, в сравнении с предыдущей версией среды разработки.

\subsubsection{Операционная система}

Ввиду того, что информационная система является веб-ориентированной, то для запуска клиентской части необходима любая операционная система, поддерживающая приложения-браузеры.

Для выбора операционной системы, на которой будет выполняться серверная часть, рассмотрим две современные операционные системы, широко использованные для размещения веб-серверов: Microsoft Windows и GNU/Linux~\cite{sunHosting}.

Как известно, обе операционные системы отлично справляются с высокими нагрузками пользователей, имеют большое количество прикладного программного обеспечения, полную и понятную документацию.
Также обе операционные системы имеют собственные реализации стека сетевых протоколов модели OSI, что вносит некоторое разнообразие в механизм обработки запроса от клиентов.
Но, в любом случае, эти различия не сильно сказываются на производительности веб-сервера.
Основную роль в выборе операционной системы сыграл пункт 1.5.3 технического задания, по которому система должна корректно запускаться и функционировать на ОС Windows Server 2008 R2.

В дополнение к выбору именно этой операционной системы можно отметить, что существует отличие в наборе инструментария для администрирования и размещения веб-проектов на основе технологий .NET.
В Microsoft Windows эти инструментарии легко подключаются и настраиваются.
В GNU/Linux также присутствуют свободные аналоги, но они не претендуют на полноту и огромную практическую значимость.

Таким образом, для разворачивания серверной части информационной системы была выбрана ОС Microsoft Windows Server 2008 R2.

\subsubsection{Средство функционального моделирования}

При разработке программного проекта рекомендуемым процессом является описание бизнес-процессов, автоматизацию которых необходимо проводить.
Для этого следует использовать программное обеспечение, обеспечивающее функциональное моделирование.

Для описания процессов предметной области была выбрана диаграмма IDEF0.

Прикладных программ, реализующих разработку данной диаграммы, множество, но они либо скудны в функциональности, устарели или платные.

Была выбрана программа Ramus Educational, которая является бесплатным аналогом бывшей BPwin (сейчас AllFusion Process Modeller).
Данная программа поддерживает создание диаграмм IDEF0, DFD, а также работу с классификаторами.
При разработке выпускной квалификационной работы этого достаточно.
Помимо этого данная программа корректно запускается на современных версиях операционной системы Microsoft Windows в отличие от устаревшего BPwin.

\subsubsection{Средство информационного моделирования}

Разработка информационной модели также необходима, как и разработка функциональной модели. Для описания такой модели была выбрана диаграмма IDEF1X.

Для разработки информационной модели была выбрана программа ERwin 7, так как уже существует лицензия на её использование.
Аналогом данной программы является ER Constructor, разрабатываемый на кафедре ИВК УлГТУ.
Но огромным недостатком данного программного обеспечения является нелогичное рисование связей между сущностями при их перемещении.
Из недостатков Erwin 7 можно отметить отсутствие поддержки последних версий операционной системы Microsoft Windows (8, 8.1 и более новых), так как последняя версия программы вышла в 2007 году.
Это отсутствие ощущается при сворачивании-разворачивании окна с программой: при этих действиях иногда модель перестаёт перерисовывать, и это необходимо делать самостоятельно при помощи соответствующей команды в меню.

\subsubsection{Вспомогательное программное обеспечение}

В дополнение к изложенным выше инструментальным средствам, для разработки информационной системы было использовано следующее программное обеспечение:

\begin{easylist}
& система управления версиями. Выбор между GIT и SVN ввиду простоты последнего был сделан в пользу SVN;
& графический пакет. Был выбран GIMP в силу бесплатности и широких возможностей. Аналогами являются Adobe Photoshop и Paint.NET;
& веб-сервер. При развёртке проекта на C\#, доступны два веб-сервера: IIS и IIS Express. Ввиду того, что функциональность последнего вполне достаточна для разработки и отладки разрабатываемой системы, был выбран он;
& система учёта задач и ошибок. Был использован Redmine из-за того, что он уже на протяжении трёх лет используется у заказчика информационной системы. Аналогами являются Jira и Bugzilla.
\end{easylist}

\subsection{Разработка прикладного программного обеспечения}

\subsubsection{Структура прикладного программного обеспечения}

Система состоит из двух подсистем: <<АРМ подрядной организации>>, делимый на модули открытой части и личного кабинета, и <<АРМ РОКР>>, в котором можно выделить модули отбора подрядчиков, продведения конкурсов и учёта договоров капитального ремонта.
Данное дробление системы на подсистемы и модули было выбрано ввиду явного разделения функций системы согласно техническому заданию.

Спецификация подсистемы <<АРМ подрядной организации>> представлена в таблице~\ref{tab:software-specArmContractor}.

\begin{footnotesize}
\begin{longtable}[h]{|p{0.05\textwidth}|p{0.3\textwidth}|p{0.55\textwidth}|}
	\caption{\label{tab:software-specArmContractor}Спецификация подсистемы <<АРМ подрядной организации>>} \\
	\hline
		\thead{№} & \thead{Название компонента} & \thead{Описание} \\
	\hline
		\theadnum{1} & \theadnum{2} & \theadnum{3} \\
	\hline \endfirsthead
	\hline
		 \theadnum{1} & \theadnum{2} & \theadnum{3} \\
	\hline \endhead
	\multicolumn{3}{|c|}{\textbf{Модули}} \\ \hline
	1 & Open & Открытая часть портала, доступная любому пользователю. \\ \hline
	2 & Personal & Личный кабинет подрядной организации. \\ \hline
	\multicolumn{3}{|c|}{\textbf{Подключаемые компоненты}} \\ \hline
	1 & AIS.HM.Model & Модель данных ИС <<Объектовый учёт>>. \\ \hline
	2 & KendoUI & Библиотека графического интерфейса. Используется для отображение таблиц и фильтров. \\ \hline
\end{longtable}
\end{footnotesize}

Спецификация подсистемы <<АРМ РОКР>> представлена в таблице~\ref{tab:software-specArmOperator}.

\begin{footnotesize}
\begin{longtable}[h]{|p{0.05\textwidth}|p{0.3\textwidth}|p{0.55\textwidth}|}
	\caption{\label{tab:software-specArmOperator}Спецификация подсистемы <<АРМ РОКР>>} \\
	\hline
		\thead{№} & \thead{Название компонента} & \thead{Описание} \\
	\hline
		\theadnum{1} & \theadnum{2} & \theadnum{3} \\
	\hline \endfirsthead
	\hline
		 \theadnum{1} & \theadnum{2} & \theadnum{3} \\
	\hline \endhead
	\multicolumn{3}{|c|}{\textbf{Модули}} \\ \hline
	1 & ContractorRequest & Модуль отбора подрядчиков. \\ \hline
	2 & Contest & Модуль проведения конкурсов. \\ \hline
	2 & ContractOnWork & Модуль учёта договоров на капитальный ремонт. \\ \hline
	\multicolumn{3}{|c|}{\textbf{Подключаемые компоненты}} \\ \hline
	1 & AIS.HM.Model & Модель данных ИС <<Объектовый учёт>>. \\ \hline
\end{longtable}
\end{footnotesize}

\subsubsection{Подсистема <<АРМ подрядной организации --- открытая часть>>}

Спецификация подсистемы указана в таблице~\ref{tab:software-specArmContractorOpen}.

\begin{footnotesize}
\begin{longtable}[h]{|p{0.05\textwidth}|p{0.45\textwidth}|p{0.4\textwidth}|}
	\caption{\label{tab:software-specArmContractorOpen}Спецификация модуля Open} \\
	\hline
		\thead{№} & \thead{Название и тип элемента} & \thead{Описание} \\
	\hline
		\theadnum{1} & \theadnum{2} & \theadnum{3} \\
	\hline \endfirsthead
	\hline
		 \theadnum{1} & \theadnum{2} & \theadnum{3} \\
	\hline \endhead
	\multicolumn{3}{|c|}{\textbf{Модули}} \\ \hline
	1 & 1 & 1. \\ \hline
	\multicolumn{3}{|c|}{\textbf{Подключаемые компоненты}} \\ \hline
	1 & 2 & 2 \\ \hline
\end{longtable}
\end{footnotesize}

\subsubsection{Подсистема <<АРМ подрядной организации --- личный кабинет>>}

Work in process.

\subsubsection{Программный модуль <<АРМ РОКР --- Отбор подрядчиков>>}

Work in process.

\subsubsection{Программный модуль <<АРМ РОКР --- Проведение конкурсов>>}

Work in process.

\subsubsection{Программный модуль <<АРМ РОКР --- Учёт договора на КР>>}

Work in process.

\subsection{Разработка инструментального средства тестирования}

Work in process.

\subsection{Особенности реализации, эксплуатации и сопровождения системы}

Work in process.

\subsection{Интерфейс пользователя с системой}

\subsubsection{Модели и технологии взаимодействия пользователя с системой}

Work in process.

\subsubsection{Руководство пользователя}

\point{Требования к условиям эксплуатации}

Work in process.

\point{Инсталляция и особенности работы}

Work in process.

\point{Порядок и особенности работы}

Work in process.

\point{Исключительные ситуации и их обработка}

Work in process.

\clearpage
\newpage