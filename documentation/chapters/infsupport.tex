\section{Информационное обеспечение системы}

\subsection{Выбор средств управления данными}

Из-за сложности информационной системы была выбрана реляционная база данных для организации хранения и обработки данных.

Рассмотрим вычисление суммы лота.
Она формируемся как сумма всех смет, которые прикреплены к лоту.
Сумма сметы определяется суммой стоимостей ремонта элементов сметы.
При использовании реляционной модели доступа к данным достаточно написать два вложенных запроса, чтобы вычислить сумму лота.
При файловой модели доступа необходимо постоянно производить многочисленные операции чтения из файловой системы, что накладывает ограничения на производительность системы в целом.
Ввиду того, что система распространяется на субъект федерации, данный критерий становится критичным.

Используемая при разработке информационной системы платформа .NET поддерживает множество транспортов данных от различных СУБД до прикладной программы.

Согласно требованиям к реляционной СУБД, обозначенным в п.1.5.2 технического задания, и лицензиям заказчика, выбор предстоит сделать между Microsoft SQL Server и Oracle Database.

Обе СУБД обладают полной документацией и работают под управлением операционной системы Microsoft Windows Server 2008 R2.
Однако для простоты разработки и дальнейшей поддержки был выбран Microsoft SQL Server, так как данная система управления базами данных работает на той же платформе .NET 4 или .NET 4.5, что и разрабатываемая информационная система.

Корпорация Microsoft большое внимание уделяет нумерации серверных компонентов.
Таким образом, при использовании операционной системы версии 2008 R2 необходимо выбрать СУБД версии 2008 R2.
Этим обеспечивается полная совместимость компонентов операционной системы и системы управления базами данных.

\subsection{Проектирование базы данных}

\subsubsection{Логическая модель данных}

Work in process.

\subsubsection{Физическая модель данных}

Work in process.

\subsubsection{Проектирование реализации}

Work in process.

\subsection{Проектирование реализации}

Work in process.

\subsection{Организация сбора, передачи, обработки и выдачи информации}

Work in process.

\clearpage
\newpage