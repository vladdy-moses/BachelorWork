\documentclass[a4paper,utf8]{eskdtext}

\newcommand{\No}{\textnumero}

% таблицы
\usepackage{longtable} 
\usepackage{multirow}

% шрифты
\usepackage{polyglossia}
\setdefaultlanguage{russian}

% борьба с "переносами" английских слов
%\usepackage{ucharclasses}
\usepackage{microtype} % better management of overfulls
\setotherlanguages{english}
% \setTransitionsForLatin{\begingroup\hyphenrules{english}}{\endgroup}

\usepackage{xecyr}
\usepackage{textcase} % для uppercase
\usepackage{amssymb,amsfonts,amsmath,amsthm} % математика
\usepackage{enumerate} % перечислялки
\usepackage{pdfpages} % для подключения pdf
% Основной шрифт
\setmainfont[Mapping=tex-text]{Times New Roman}
% Моноширинный шрифт
%\setmonofont[Scale=MatchLowercase]{Courier New}
% Стандартные сочетания символов ---, --, << >> и т.п.
\defaultfontfeatures{Mapping=tex-text}
% Переносы в русских текстах
\newfontfamily\russianfont{Times New Roman}
% Для переноса составных слов
\XeTeXinterchartokenstate=1
\XeTeXcharclass `\- 24
\XeTeXinterchartoks 24 0 ={\hskip0pt}
\XeTeXinterchartoks 0 24 ={\nobreak}

%\setmainfont[Mapping=tex-text]{GOST type A}

% Устанавливаем шрифты "ГОСТ А" и "ГОСТ Б" в рамках ЕСКД
%\renewcommanf{\ESKDfontShape}{\textit}
%\renewcommand{\ESKDfontShape}{\fontspec[BoldFont={Mipgost}]{Mipgost}}
\renewcommand{\ESKDfontShape}{\fontspec{Mipgost}\itshape\slshape}


% отступы
% \ESKDsectSkip{section}{4em}{4em}
% \ESKDsectSkip{subsection}{3em}{2em}

\ESKDsectStyle{section}{\bfseries\Large\centering\MakeUppercase}

% lists
\usepackage[ampersand]{easylist}
\ListProperties(Hide=100, Hang=true, Progressive=1.5cm, Style*=-- , Style2*=$\bullet$, Style3*=$\circ$, Style4*=\tiny$\blacksquare$ )

\usepackage{ulem} % для умных подчёркиваний

% информация о работе
% === Фамилии ===
\newcommand{\AuthorName}{Моисеев В.В.}
\newcommand{\AuthorNameFullOf}{Моисеева Владислава Валерьевича}

\newcommand{\ManagerName}{Родионов В.В.}
\newcommand{\ManagerNameFull}{Родионов Виктор Викторович}

\newcommand{\ManagerNameEconomics}{Рыбкина М.В.}
\newcommand{\ManagerNameEcology}{Куклев В.А.}

\newcommand{\ControllerName}{Докторов А.Е.}

\newcommand{\ReviewerName}{КТО ЖЕ ТЫ?!}

%+add рецензента

% === Названия, номера ===
\newcommand{\AuthorGroup}{ИСТбд-41}
\newcommand{\WorkName}{Информационный портал для организации подрядных работ по капитальному ремонту}
\newcommand{\WorkNumber}{ДП-УлГТУ-230400162-11/300-2015}

% === Мелочь ===
\newcommand{\WorkYear}{2015}
\newcommand{\WorkPlace}{Ульяновск}

\ESKDdepartment{МИНИСТЕРСТВО ОБРАЗОВАНИЯ И НАУКИ РОССИЙСКОЙ ФЕДЕРАЦИИ}
\ESKDcompany{федеральное государственное образовательное учреждение высшего профессионального образования \\ 
Ульяновский Государственный Технический Университет}

\ESKDdocName{Пояснительная записка} % (графа 1)

% графы 11
\ESKDauthor{\AuthorName}
\author{\AuthorName}
\ESKDchecker{\ManagerName}
\renewcommand{\ESKDcolumnXfIVname}{Реценз.}
\ESKDcolumnXIfIV{\ReviewerName}
\ESKDapprovedBy{\ControllerName}
\ESKDnormContr{\NormControllerName}

\ESKDletter{У}{Р}{} % учебный реальный (графа 4)
\ESKDgroup{\AuthorGroup} % учебная граппа (графа 9)
\ESKDsignature{\WorkNumber~ПЗ} % обозначение документа (по п.6.1.1) 

\ESKDdefaultTitleStyle{empty}

\renewcommand{\ESKDtheTitleFieldX}{Ульяновск, 2015}

% списочки
\usepackage[shortlabels]{enumitem}
%\setlist[enumerate]{nosep}
\setlist[enumerate]{nosep,label=-}
\AddEnumerateCounter{\Asbuk}{\@Asbuk}{\CYRM}
\AddEnumerateCounter{\asbuk}{\@asbuk}{\cyrm}
% \let\oldenumerate\enumerate
% \renewcommand{\enumerate}{
  % \oldenumerate
  % \setlength{\itemsep}{1pt}
  % \setlength{\parskip}{0pt}
  % \setlength{\parsep}{0pt}
% }