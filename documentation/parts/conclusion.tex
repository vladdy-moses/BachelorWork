\section*{Заключение}
\addcontentsline{toc}{section}{Заключение}

Выпускная квалификационная работа была выполнена в необходимом и достаточном объёме.
Были использованы средства информационного, функционального моделирования, интегрированная среда разработки, средства управления данными и иное программное обеспечение.

Информационная система сопровождается группой компаний ООО <<АИС Город>>.
Она внедрена и активно используется во Владимирской и Ульяновской областях.
Есть упоминание веб-портала подрядных организаций в постановлении администрации Владимирской области от 16.05.2014 N 491 (ред. от 01.08.2014) <<О порядке привлечения региональным оператором подрядных организаций для оказания услуг и (или) выполнения работ по капитальному ремонту общего имущества в многоквартирных домах>>.

Автор выпускной квалификационной работы представлял систему на студенческой научно-технической конференции, проводимой в УлГТУ весной 2015 года, где получил дипломы I и II степени.
Комиссия особенно отметила актуальность системы и её внедрение.

Также был написан ряд статей по применению современных технологий при разработке веб-ориентированных информационных систем.
Были рассмотрены препроцессор LESS, проведён сравнительный анализ различных систем доступа к данным, описаны преимущества и недостатки метода распространения программного обеспечения SaaS в жилищно-коммунальном хозяйстве.

Информационная система имеет перспективы развития как площадки отбора подрядных организаций не только для проведения капитального ремонта жилищного фонда субъекта федерации, но и для составления проектно-сметной документации, а также для любых других конкурсных процессов, проходящих в сфере жилищно коммунального хозяйства.

\clearpage
\newpage