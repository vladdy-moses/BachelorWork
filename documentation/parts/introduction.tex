\section*{Введение}
\addcontentsline{toc}{section}{Введение}

\subsection{Краткое описание предметной области}

Work in process.

\subsection{Анализ используемых источников}

Весь список использованных при написании выпускной квалификационной работы источников можно разбить на несколько групп:

\begin{enumerate}
	\item источники по предметной области;
	\item источники, связанные с инструментальными средствами;
	\item ГОСТы;
	\item учебные пособия университета.
\end{enumerate}

К группе источников, описывающих предметную область, можно отнести федеральные законы и иные нормативно-правовые акты в сфере жилищно-коммунального хозяйства, а также комментарии к определениям <<капитальный ремонт>> и <<региональный оператор капитального ремонта>>.

В жилищном кодексе Российской Федерации содержатся основные понятия, использованные в жилищно-коммунальном хозяйстве, а также процессы данной области экономики.
В дипломной работе данный закон используется для определения понятий жилого фонда и регионального оператора, а также функций последнего.

В статье <<Капитальный ремонт>> даётся определение данному понятию.

В технических требованиях на создание <<Единой информационно-аналитической информационной системы жилищно-коммунального хозяйства Московской области>> систематизированы механизмы автоматизации всего жилищно-коммунального хозяйства.
Особый интерес представляет разъяснение автоматизации капитального ремонта, что непосредственно относится к предметной области данной работы.

На сайте администрации Санкт-Петербурга описываются конкурсы на проведение капитального ремонта жилого фонда города на 2015 год, принципы проведения таких конкурсов.
На сайте даны ссылки на сайт регионального оператора капитального ремонта Санкт-Петербурга.

В статье <<Портал поставщиков>> рассказывается про аналог разрабатываемой системы, уже внедрённой в Москве.
Рассказывается также про возможности веб-сайта, его интеграции с единой системой торгов города Москвы, а также описываются базовые процессы, которые можно осуществить на сайте.

В тексте статьи на сайте <<Гранит-центр>> идёт прямая ссылка на федеральный закон № 94-ФЗ, который уже потерял актуальность.
Взамен этому закону сейчас применяется № 44-ФЗ <<О контрактной системе в сфере закупок товаров, работ, услуг для обеспечения государственных и муниципальных нужд>>.
Он также был изучен в ключе применимости к дипломной работе.
Источником была выбрана система <<КонсультантПлюс>>.
В тексте закона в статье 1 описывается сфера применения настоящего закона. Дипломная работа полностью удовлетворяет пунктам 1.2, 1.4 и 1.5 статьи 1.

К источникам, связанным с инструментальными средствами, можно определить книги Фленова, Троелсена и Чертовского.

В работе <<Библия C\#>> описываются основы программирования на языке C\#, а также рассматриваются примеры реализации простейших приложений.
В этой книге описываются стандарты написания исходного кода на языке программирования C\#, а также возможности IDE Microsoft VisualStudio.
Данные рекомендации будут использованы при реализации информационной системы.

В монументальной работе <<Язык программирования C\# 2010 и платформа .NET 4>> описываются как основы языка и платформы, на экосистеме которой он функционирует, так и более глубокие механизмы взаимодействия языка и фреймворка.
В книге указано практическое применение технологий Microsoft, ADO.NET и ASP.NET, которые были применены в реализации информационной системы.

В книге <<Базы и банки данных: Учебное пособие>> описываются основы реляционной алгебры, теории реляционных баз данных, а также языка определения данных (ЯОД) и языка манипулирования данными (ЯМД).
Книга помогает получить общую картину устройства языка SQL, особенности современных диалектов языка реляционных баз данных, систем управления базами данных, которые будут использованы при реализации информационной системы.

К ГОСТам были отнесены непосредственно те стандарты, использование которых необходимо как при разработке информационной системы, так и при подготовке пояснительной записки дипломной работы.

ГОСТ 19.701-90 относится к группе стандартов оформления программной документации, принятой на территории России, а также других стран, признающих данный стандарт.
В этом ГОСТ описываются принципы изображения блок-схем, которые понадобятся при описании математического обеспечения информационной системы.
Стандарт в полной мере раскрывает алгоритм построения блок-схем, принятые в них обозначения, применимость последних в разных видах схем, а также содержит примеры.

В ГОСТ 34.601-90 описываются основные этапы и стадии разработки информационной системы.
Изучение этого стандарта необходимо для правильной оценки экономической составляющей информационной системы, а также при оценке сроков разработки.
Знание стадий разработки информационных систем помогает построить данный процесс таким образом, чтобы обезопасить себя от нехватки времени в последние дни перед приёмкой системы.

В ГОСТ 7.1-2003 описываются стандарты оформления библиографического списка, что также требуется изучить при составлении пояснительной записки дипломной работы.
Система не может разрабатываться без вспомогательных источников, а правильное оформление их списка позволяет оценить качество и трудоёмкость работы.

При подготовке текста пояснительной записки дипломной работы, а также при организации действий дипломника необходимо помимо государственных стандартов учитывать локальные стандарты университета, в котором работа будет защищаться.
Выпускающая кафедра <<Измерительно-вычислительные комплексы>> УлГТУ подготовила ряд учебных пособий и методических указаний для упрощения работы над дипломной работой.

В работе <<Дипломное проектирование: учебно-методическое пособие для студентов специальности 23020165 “Информационные системы и технологии”>> разрешаются основные трудности выпускника при подходе к началу дипломного проектирования.
Также в пособии указаны этапы работы дипломника, принятые на кафедре стандарты оформления пояснительной записки и описание отдельных её частей.
К примеру, указано несколько шаблонов оформления главы <<Техническое задание>> и дано полное описание к пунктам одного из шаблонов. 
Данная работа является основным методическим указанием при дипломном проектировании.

В учебном пособии <<Технико-экономический анализ инженерного проекта>> описываются методы и принципы оценки экономической эффективности использования инженерного проекта.
В рамках дипломной работы под инженерным проектом принято считать разрабатываемый программный продукт, а также перечень проектной документации, необходимый на всех этапах жизненного цикла программного продукта.
Данное учебное пособие помогает рассчитать расходы, прибыль и экономическую эффективность разрабатываемой информационной системы, что отражено в соответствующей главе пояснительной записки.

\clearpage
\newpage