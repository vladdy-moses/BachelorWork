\section*{Введение}
\addcontentsline{toc}{section}{Введение}

\subsection*{Краткое описание предметной области}
\addcontentsline{toc}{subsection}{Краткое описание предметной области}

Информатизация жилищно-коммунального хозяйства – ответственная и сложная работа.
Она включает в себя, прежде всего, повышение прозрачности ЖКХ, уменьшение бюрократии в данной сфере экономики, а также приведение деятельности различных должностных лиц, организаций и органов власти к действующему законодательству России.

В жилищно-коммунальном хозяйстве существует множество областей автоматизации.
Одной из них является процесс проведения капитального ремонта.
Капитальный ремонт в ЖКХ представляет собой воспроизводство основных характеристик жилого фонда путём замены отдельных частей зданий \cite{economicdict}.
Перед проведением капитального ремонта составляется план капитального ремонта на несколько лет, что входит в полномочия регионального оператора капитального ремонта отдельного субъекта России \cite[ст. 180]{fz188}.
Так как региональный оператор не может самостоятельно делать ремонт, а лишь следит за его выполнением, он вправе выбирать организации, которые будут заниматься непосредственно ремонтом жилого фонда.
Эти организации для единого стиля описания назовём подрядными или подрядчиками.
В их ведение входит выполнение ремонта здания, зданий или отдельных инженерных конструкций \cite{analog_spb}.

Данная дипломная работа ставит задачу автоматизировать процесс регистрации подрядной организации в реестре подрядных организаций, подачи заявки на проведение капитального ремонта жилого фонда, а также автоматизировать контроль над проведением ремонта.
Также сфера автоматизации дипломной работы непосредственно коррелирует с механизмом подачи торгов на выполнение работ, что в полной мере описывается в федеральном законе № 44-ФЗ \cite{fz44}.

В данной предметной области присутствует две роли:

\begin{enumerate}
	\item подрядная организация --- та компания, которая может провести ремонт жилого фонда;
	\item региональный оператор капитального ремонта (или РОКР) --- организация, проводящая контроль содержания жилого фонда субъекта федерации \cite[ст. 178, ст.180]{fz188}.
\end{enumerate}

Заявки на капитальный ремонт подаются на лоты, которые объединяются в конкурсы на проведение капитального ремонта \cite{analog_spb}.
Например, существует конкурс от 24 февраля 2015 года. В нём содержится 9 лотов на 9 домов жилого фонда г. Санкт-Петербурга.
В этих многоквартирных домах требуется отремонтировать газовое оборудование \cite{analog_spb}.

При рассмотрении взаимодействия подрядчика и РОКР \cite{analog_msk}, можно сделать заключение, что они общаются при следующих процессах:

\begin{enumerate}
	\item подрядчик регистрируется в реестре подрядчиков регионального оператора капитального ремонта;
	\item РОКР одобряет внесение подрядчика в реестр или не одобряет, сопровождая отказ комментарием;
	\item подрядчик подаёт заявку на капитальный ремонт;
	\item региональный оператор капитального ремонта при наступлении дня вскрытия конвертов выбирает лучшую заявку;
	\item также РОКР направляет сообщение выигравшему подрядчику;
	\item подрядчик и РОКР заключают договор о капитальном ремонте;
	\item подрядчик отчитывается перед региональным оператором о плановых сроках и стоимости работ;
	\item РОКР одобряет плановые показатели или не одобряет, сопровождая отказ комментарием;
	\item подрядная организация после проведения ремонта заполняет фактические сроки и стоимости работ, сопровождая ответ сметами КС-2 и КС-3;
	\item региональный оператор капитального ремонта либо одобряет фактические показатели, либо не одобряет, сопровождая отказ комментарием.
\end{enumerate}

Как можно заметить, практически все точки взаимодействия подрядчика и регионального оператора капитального ремонта можно перенести в информационную среду.
При этом уменьшится время, затраченное на издержки бумажного общения.

Также стоит отметить, что у подрядчика может быть несколько ролей сотрудников:

\begin{enumerate}
	\item ответственное лицо за поиск подходящих конкурсов, где подрядная организация может участвовать;
	\item ответственное лицо за подачу заявки на проведение капитального ремонта жилого фонда;
	\item директор организации, имеющий полный доступ к информации о собственной компании.
\end{enumerate}

У регионального оператора в контексте данной предметной области также определены следующие роли:

\begin{enumerate}
	\item сметчик, описывающий сметы на капитальный ремонт;
	\item ответственное лицо за размещение и публикацию конкурсов;
	\item ответственное лицо за проведение конкурса, вскрытие конвертов;
	\item ответственный за надзор за ходом капитального ремонта.
\end{enumerate}

Также в различных субъектах федерации действуют различные правила по публикации информации о конкурсах, но в целом они сводятся к тому, что информация любом проводимом конкурсе должна быть где-либо размещена.

Доподлинно известно то, что парк персональных компьютеров персонала, работающего в данной предметной области, достаточно разнообразен и не подчиняется чётким требованиям, логичнее всего из списка современных технологий разработки приложений к персональным компьютерам использовать при построении информационной системы такую среду, как Интернет.

На данный момент существует множество технологий для построения веб-ориентированных информационных систем, но так как на данный момент широкое распространение в информатизации ЖКХ получили технологии от корпорации Microsoft, для более простой интеграции с существующими системами логичнее использовать схожие технологии.
Платформа .NET позволяет построить информационную систему от СУБД до веб-сервера полностью на своей основе \cite{troelsen,flenov}.

\subsection*{Анализ используемых источников}
\addcontentsline{toc}{subsection}{Анализ используемых источников}

Весь список использованных при написании выпускной квалификационной работы источников можно разбить на несколько групп:

\begin{enumerate}
	\item источники по предметной области;
	\item источники, связанные с инструментальными средствами;
	\item ГОСТы;
	\item учебные пособия университета.
\end{enumerate}

К группе источников, описывающих предметную область, можно отнести федеральные законы и иные нормативно-правовые акты в сфере жилищно-коммунального хозяйства, а также комментарии к определениям <<капитальный ремонт>> и <<региональный оператор капитального ремонта>>.

В жилищном кодексе Российской Федерации содержатся основные понятия, использованные в жилищно-коммунальном хозяйстве, а также процессы данной области экономики.
В дипломной работе данный закон используется для определения понятий жилого фонда и регионального оператора, а также функций последнего.

В статье <<Капитальный ремонт>> даётся определение данному понятию.

В технических требованиях на создание <<Единой информационно-аналитической информационной системы жилищно-коммунального хозяйства Московской области>> систематизированы механизмы автоматизации всего жилищно-коммунального хозяйства.
Особый интерес представляет разъяснение автоматизации капитального ремонта, что непосредственно относится к предметной области данной работы.

На сайте администрации Санкт-Петербурга описываются конкурсы на проведение капитального ремонта жилого фонда города на 2015 год, принципы проведения таких конкурсов.
На сайте даны ссылки на сайт регионального оператора капитального ремонта Санкт-Петербурга.

В статье <<Портал поставщиков>> рассказывается про аналог разрабатываемой системы, уже внедрённой в Москве.
Рассказывается также про возможности веб-сайта, его интеграции с единой системой торгов города Москвы, а также описываются базовые процессы, которые можно осуществить на сайте.

В тексте статьи на сайте <<Гранит-центр>> идёт прямая ссылка на федеральный закон № 94-ФЗ, который уже потерял актуальность.
Взамен этому закону сейчас применяется № 44-ФЗ <<О контрактной системе в сфере закупок товаров, работ, услуг для обеспечения государственных и муниципальных нужд>>.
Он также был изучен в ключе применимости к дипломной работе.
Источником была выбрана система <<КонсультантПлюс>>.
В тексте закона в статье 1 описывается сфера применения настоящего закона. Дипломная работа полностью удовлетворяет пунктам 1.2, 1.4 и 1.5 статьи 1.

К источникам, связанным с инструментальными средствами, можно определить книги Фленова, Троелсена и Чертовского.

В работе <<Библия C\#>> описываются основы программирования на языке C\#, а также рассматриваются примеры реализации простейших приложений.
В этой книге описываются стандарты написания исходного кода на языке программирования C\#, а также возможности IDE Microsoft VisualStudio.
Данные рекомендации будут использованы при реализации информационной системы.

В монументальной работе <<Язык программирования C\# 2010 и платформа .NET 4>> описываются как основы языка и платформы, на экосистеме которой он функционирует, так и более глубокие механизмы взаимодействия языка и фреймворка.
В книге указано практическое применение технологий Microsoft, ADO.NET и ASP.NET, которые были применены в реализации информационной системы.

В книге <<Базы и банки данных: Учебное пособие>> описываются основы реляционной алгебры, теории реляционных баз данных, а также языка определения данных (ЯОД) и языка манипулирования данными (ЯМД).
Книга помогает получить общую картину устройства языка SQL, особенности современных диалектов языка реляционных баз данных, систем управления базами данных, которые будут использованы при реализации информационной системы.

К ГОСТам были отнесены непосредственно те стандарты, использование которых необходимо как при разработке информационной системы, так и при подготовке пояснительной записки дипломной работы.

ГОСТ 19.701-90 относится к группе стандартов оформления программной документации, принятой на территории России, а также других стран, признающих данный стандарт.
В этом ГОСТ описываются принципы изображения блок-схем, которые понадобятся при описании математического обеспечения информационной системы.
Стандарт в полной мере раскрывает алгоритм построения блок-схем, принятые в них обозначения, применимость последних в разных видах схем, а также содержит примеры.

В ГОСТ 34.601-90 описываются основные этапы и стадии разработки информационной системы.
Изучение этого стандарта необходимо для правильной оценки экономической составляющей информационной системы, а также при оценке сроков разработки.
Знание стадий разработки информационных систем помогает построить данный процесс таким образом, чтобы обезопасить себя от нехватки времени в последние дни перед приёмкой системы.

В ГОСТ 7.1-2003 описываются стандарты оформления библиографического списка, что также требуется изучить при составлении пояснительной записки дипломной работы.
Система не может разрабатываться без вспомогательных источников, а правильное оформление их списка позволяет оценить качество и трудоёмкость работы.

При подготовке текста пояснительной записки дипломной работы, а также при организации действий дипломника необходимо помимо государственных стандартов учитывать локальные стандарты университета, в котором работа будет защищаться.
Выпускающая кафедра <<Измерительно-вычислительные комплексы>> УлГТУ подготовила ряд учебных пособий и методических указаний для упрощения работы над дипломной работой.

В работе <<Дипломное проектирование: учебно-методическое пособие для студентов специальности 23020165 “Информационные системы и технологии”>> разрешаются основные трудности выпускника при подходе к началу дипломного проектирования.
Также в пособии указаны этапы работы дипломника, принятые на кафедре стандарты оформления пояснительной записки и описание отдельных её частей.
К примеру, указано несколько шаблонов оформления главы <<Техническое задание>> и дано полное описание к пунктам одного из шаблонов. 
Данная работа является основным методическим указанием при дипломном проектировании.

В учебном пособии <<Технико-экономический анализ инженерного проекта>> описываются методы и принципы оценки экономической эффективности использования инженерного проекта.
В рамках дипломной работы под инженерным проектом принято считать разрабатываемый программный продукт, а также перечень проектной документации, необходимый на всех этапах жизненного цикла программного продукта.
Данное учебное пособие помогает рассчитать расходы, прибыль и экономическую эффективность разрабатываемой информационной системы, что отражено в соответствующей главе пояснительной записки.

\clearpage
\newpage